\section{The surveys }
Our data was obtained through surveys that were conducted among two pools of
participants. In August 2018, we asked economic experts from the World
Economic Survey (WES) for their opinion on perceptions about the financial
entrenchment following the European debt crisis. The WES is a quarterly
survey conducted by the ifo Institute. The WES sample includes 517
participants from EU member countries, among them 90 experts from program countries.
The survey regularly includes questions about overall economic development
from European and non-European experts such as economic growth and
inflation. The WES in August 2018 included a special module with a set of
questions on the European debt crisis (see the Appendix for the
questionnaire). \footnote{Scholars use the WES to include special modules. See, for example, \cite{mosler}.}

Our second pool of participants was recruited through the website
prolific.co. In contrast to other crowdsourcing platforms such as Mturk,
Prolific is a platform specifically designed to recruit participants for
academic research \footnote{\cite{Peer} demonstrate that participants from
the platform prolific perform better than participants from other
crowdsourcing platforms.}. In exchange for their participation in surveys or
experiments, participants receive a financial reward. The survey was
distributed to 1702 participants in August 2019, 498 of these participants
came from program countries. To ensure that our participants had an
opportunity to actively remember the events during the European debt crisis
we restrict our sample to include only participants older than 25.

The precisely same survey questions were used in both surveys, and in
unchanged ordering. The appendix shows the country composition of survey
participants for both samples.\footnote{%
The WES sample and the prolific sample naturally differ along various
dimensions. The differences are largely due to the expert status of members
of the WES sample. The majority of participants from the prolific sample,
around 65 percent are younger than 35, whereas the majority of participants
from the WES sample are between 35 and 55. Participants from the WES sample
are also have a higher level of education than participants from the
prolific sample, 60 percent of participants hold a PhD. Nonetheless the
majority of participants from the prolific sample have completed tertiary
education. More than 80 percent of WES participants are male, whereas in the
prolific sample there is an equal share of men and women.} We refer to the
WES sample as the expert sample and to the prolific sample as the non-expert
sample. We are interested in whether the survey participants show a
nation-serving bias in their answers, and whether there is a structural
difference between replies by ordinary citizens and by experts.