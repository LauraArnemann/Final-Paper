\clearpage
\section{Results}
We present the results of our empirical analysis for the sample of experts (the WES sample) and the sample of non-experts (the prolific sample).  We estimate the following model by binary logit. 
\begin{equation*}
    Y_{ijc}= \alpha_{j}+ \beta *D_{c} + \gamma*X_{i}
\end{equation*}
Where $Y_{ijc}$ denotes the response of individual $i$ from country $c$ to question $j$ we will control for individual characteristics $X_{i}$ such as age, level of education, gender and employment status (affiliation for the expert sample). The dummy variable $D_{c}$ indicates whether the respondents nationality is Greek, Cypriotic, Spanish, Irish or Portuguese with the coefficient $\beta$ measuring the divergence in the answers of participants from program and non-program countries.For questions in which participants were asked to state their level of agreement we estimate the effect of being from a program country on the likelihood to strongly or slightly agree. For questions in which participants are asked to name the responsible party, we estimate the effect of being from a program country on the likelihood to state "Lender countries". \\ \\
\textbf{Baseline Results} 
When asked about the intentions of lender countries to initiate the credit relationship our hypothesis is verified among our non-expert participants.
 In the non-expert sample citizens from program countries are 10.9 percentage points more likely to agree that lender countries wanted to impose institutional change upon the borrower countries. They are 13.1 percentage points less likely to agree that the lender countries wanted to help the borrowing countries. However, there is no significant effect on agreeing that lender countries merely wanted to avoid a crisis at home. In the expert sample there are no statistically significant differences in the assessments of experts from program and non-program countries. Contrasting our hypothesis,  experts from program countries are more likely to agree that the lender countries wanted to help the borrowing countries. This effect does not turn out to be statistically significant.  \\ 
 \begin{figure}[H]
\begin{center}
     \caption{Intentions of the lender countries}
    
     \includegraphics[scale=0.8]{Question2_base.pdf}
     \label{fig:my_label}
      \end{center}
      \tiny
     \tablenotes{The exact wording of the questions were the following.  Question 2.1: The lender countries wanted to help the borrowing countries Question 2.2: The lender countries wanted to help themselves avoid a crisis at home Question 2.3: The lender countries wanted to impose institutional change upon the borrower countries  }
\end{figure}
 We continue to ask the participants about the emotions the rescue program evoked among citizens from the borrower and lender countries. The hypothesis that lenders have a blind spot regarding the feelings of citizens from borrower countries is confirmed. Participants from program countries in the non-expert sample are 9.2 and 9.3 percentage points more likely to agree that the rescue experience made them feel guilty and inferior. Further, the probability to agree to "The rescue experience made many citizens in the borrower countries feel exploited" increases by 16.9 percentage points among citizens from program countries. All effects are statistically significant at the 1 percent level. Among the sample of economic experts the effect sizes are smaller and not significantly different from zero. \\
 Interestingly, we do not find that borrowers have a blind spot regarding the emotions the program evoked among lender countries. There are only small differences in the likelihood to agree with certain statements between citizens from program and non-program countries which are not significant. There is no difference in evaluating the effect of the rescue program on the friendship between citizens. The results are similar in the expert and non-expert sample. \\

 We further predicted that borrower countries will be more confident that they will be able to repay outstanding debt. 
 When asked whether Greece \footnote{We only ask about Greece, since Greece remains the only country that has not repaid it's debt} will be capable of fully paying back it's debt, citizens from program countries show a 13.5 percentage point higher likelihood to agree in the non-expert sample and a 12.2 percentage point higher likelihood in the expert sample. This effect is again significant at the 1 percent level. \\
 \begin{figure} [h!]
    \begin{center}
    \caption{Sentiments of the borrower countries}
    \includegraphics[scale=0.8]{Question5_1_base.pdf}
    \label{fig:my_label}
    \end{center}
    \tiny
    \tablenotes{The exact wording of the questions are the following. Question 5.1: The rescue experience made many citizens in the borrower countries feel guilty; Question 5.2: The rescue experience made many citizens in the borrower countries feel exploited; Question 5.3: The rescue experience made many citizens in the borrower countries feel inferior. }
\end{figure}
\begin{figure}[h!]
\begin{center}
\caption{Sentiments among lender country citizens}

    \includegraphics[scale=0.8]{Question5_2_base.pdf}
    \label{fig:my_label}
    \end{center}
    \tiny
    \tablenotes{The sign in parantheses denoted the predicted differential effect. Question 5.4: The rescue experience made many citizens in the lender countries feel exploited; Question 5.5 The rescue experience made many citizens in the lender countries feel disappointed Question 5.6: The rescue experience strengthened friendships between citizens Question 7: Greece will fully pay back it's debt}
\end{figure}
 
%  \begin{figure}
%\caption{Assessment of the emotions the program evoked among different parties}
%\centering
%\begin{minipage}{.5\textwidth}
% \includegraphics[scale=0.5]{Question5_1_base.pdf}
 %\end{minipage}%
 %\hfill
%\begin{minipage}{.5\textwidth}
%\includegraphics[scale=0.5]{Question5_2_base.pdf}
%\end{minipage}
%\end{figure}
 We proceed to evaluate who initiated and who benefited from the credit relationship. 
In the non-expert sample citizens are 13.2 percentage points more likely to state that the lender countries were the driving force behind signing the memorandum. Further, they are 28 percentage points more likely to state that the lender countries were the main beneficiaries of the program and 20.3 percentage points more likely to state that lender countries benefited from the loans to Greece. All effects are statistically significant at the one percent level. We do not find statistically significant effects in the expert sample. Interestingly, the mean difference between expert and non-experts is negative, contrary to the anticipated effect. Experts from program countries are for example 12.4 percentage points less likely to state that citizens from lender countries were the main beneficiary from the program.\\
\begin{figure}[h!] 
\begin{center}
     \caption{Who initiated and benefited from the rescue program}
     \includegraphics[scale=0.8]{Question3_4_base.pdf}
     \label{fig:my_label}
     \end{center}
     \tiny
     \tablenotes{The sign in parantheses denotes the predicted differential effect.Question 3: Who was the driving force behind signing the memorandum; Question 4: Who was the main beneficiary of the program; Question 7: Who primarily benefited from the loans to Greece}
\end{figure}

\begin{figure}[h!] 
\begin{center}
     \caption{Situation in Greece}
     \includegraphics[scale=0.8]{Question6_7_base.pdf}
     \label{fig:my_label}
     \end{center}
     \tiny
     \tablenotes{The exact wording of the question was the following.   Question 6: Who primarily benefited from the loans to Greece; Question 7: Greece will fully pay back it's debt}
\end{figure}
\clearpage
\textbf{Sample Splits and Heterogeneity Analysis}
Program and non-program countries vary along other dimensions than the program and non-program distinction. Program countries share several characteristics which are likely to influence the estimates. Countries which were affected by the European debt crisis are predominantly Southern European and are located at the periphery of the European Union (measured by distance to Brussels). Further, all these countries experienced a substantial increase in debt levels, high levels of unemployment, low GDP growth and belonged to the Eurozone. Thus we conduct sample splits along these margins to test whether these variables drive our results. \footnote{ We again estimate our model on the subsample of countries which are Southern European, belong to the Eurozone,are located at the periphery of Europe,defined as countries in which the distance of the capital to Brussels is above the median, experienced above median debt and unemployment growth and below median GDP growth during the years 2007 and 2012.} The overall findings of our analysis remain unchanged among the expert and non-expert sample.
\\
Experts and non-experts differ along various dimensions which might influence the magnitude of effect and explain the observed differential impact of belonging to a program country. Experts are typically older, have a higher level of education and might also be more mobile than non-experts. We estimate our model for the sample of experts older than 35, tertiary education and who report to be living in a different country than their country of birth. The results of the non-expert sample do not change for the sample splits along age and education, but do show some changes along the variable mobility. Hence, being exposed to a more international environment could be driving the observed differences in results between the expert and non-expert sample. Additionally, we also redefine the program variable according to beliefs of the survey participants. We now estimate the effect of believing to live in a borrower country on the outcomes of our survey. Interestingly, the effect of believing to live in a program country is smaller than the effect of actually living in a program country. In addition to the heterogeneity analysis we also conduct several robustness checks. We estimate the model by ordered and multinomial estimation, we cluster standard errors at the country level, control for multiple hypothesis testing and drop inattentive respondents from our non-expert sample. All these checks do not yield substantially different results than our baseline estimation. A detailed overview of the results of our heterogeneity analysis and robustness checks can be found in the appendix. 


