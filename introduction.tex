\section{Introduction}

This paper studies collective memories of the European debt crisis that
became salient in 2010. We explore systematic differences in collective
memories between different groups:\ memories of citizens and experts, sorted
according to whether they are in program countries or non-program countries.
The European debt crisis partitioned the countries inside the Eurozone. Some
Eurozone member states experienced fiscal and financial conditions that
dramatically worsened and were in danger of losing access to the capital
market for refinancing their public loans. The remaining countries had to
fear contagion effects and other negative spillovers of possible national
bankruptcies in the more severely affected countries.\ Five countries
(Greece, Ireland, Portugal, Spain and Cyprus) applied for, and signed a
memorandum of understanding with the member countries of the European
Monetary Union. As part of these agreements, they received financial aid or
guarantees, but also accepted supervised mandatory structural reforms.
Simplifying and abbreviating these measures taken, we refer to the program
countries as the 'borrower countries'. On the other side were the countries
that used their fiscal credibility to provide these guaranties and requested
and participated in a monitoring of the process of structural reforms. We
refer to them as 'lender countries'.

The key question we ask is whether there are systematic differences in how
citizens and experts from borrower and lender countries remember the crisis.
Will citizens from the two country groups have similar memories about who
actually applied for a memorandum of understanding? How do they remember
whether the lender or the borrower countries pushed more for it and whether
the memoranda of understanding mainly benefited the lender countries or the
borrower countries? And if there are biases along the sets of country
groups, do these differ between regular citizens and experts? We also
investigate how these programs were perceived in the respective countries
and how these crisis interventions affected the relationship between European
countries more generally.\ 

To study these questions we first considered possible divergence in the
views of experts. For this purpose we included specific questions to the
World Economic Survey (WES) panel of experts. Then we paid a larger number
of members of the general population in countries in Europe in an internet
questionnaire (https://prolific.co) to answer the same set of questions. We
find no systematic divergence in the answers between the WES experts from
borrower countries and experts from lender countries. However, significant
differences emerge in how the citizens from the two groups of countries
remember what countries signed a memorandum, why they signed it, and how
they assess the measures taken to address the crisis.

The empirical analysis of divergence of memories along the lines of borrower
countries and lender countries is motivated by several existing insights.
Memory formation has been studied in psychology, neuroscience and sociology.
Insights about the plasticity of memory\footnote{%
The physiological basics for how memory is formed, kept and reactivated have
been explored. As Dudai and Edelson (2016;\ 276)\ describe, "When the memory
is retrieved, it seems to re-enter a transient phase in which it again
becomes susceptible to the same amnesic agents that were effective in the
original consolidation window (Dudai, 2012; Nader et al., 2000; Sara,
2000)." Brain sciences hence suggest that memory enters a state of
plasticity when it is reactivated and the copy that is then stored might
differ from the one that has been activated (see Agren 2014, and Lee, Nader
and Schiller 2017).}, and a tendency to memorize in a 'self-serving' way
(see Bell and Echterhoff 2014) have been combined. If this self-serving bias
affects the process of reactivating, transforming and storing memory, it can
cause memory with a self-serving drift.

This combination of plasticity of memory and the self-serving drift has been
applied by Desz\"{o} and Loewenstein (2012) to hypothesize self-serving
memories on informal credit relationships. Using data about informal credit
relationships between relatives and friends, they found that borrowers and
lenders diverge in their memories about these credit relationships in
 ways that appear to be self-serving.\footnote{%
Borrowers and lenders partially diverge in how they recall the credit event
as such, the conditions of the loan they agreed upon, and the mutual
interactions that made the contract come about. This divergence typically
has a self-serving bias. For instance, compared to the lenders, borrowers
tend to recall less if they exerted moral pressure to receive the loan and
are more likely to recall that the credit was generously offered to them.}

The lenders and borrowers in the European souvereign debt crisis are
nations, rather than individuals, but the line of reasoning could be
similar: when analyzing the recollections of fiscal credit relationships or
loan guarantees between nations, the mental processes of the formation,
storage and reactivation of memory are still those of single individuals.
They formation of memories of individuals who belong to the same group might
follow the same logic and pyhsiological laws as in the context of private
borrowing-lending relationships. In the aggregate, individuals in borrower
countries and individuals in lender countries might recall differently which
side was the driving force for international credit relationships: the
lender-side interpretation might more often be that the borrower country was
desperately seeking help, whereas the borrower side might tend to identify
reasons why the lender-side forced the program upon the borrower country.
The two groups might also differ in their assessments about who eventually
benefited more from the conditional financial help program.\footnote{%
A large literature discusses collective memory, in difference to the memory
of individuals. Olick (1999) provides a lucid survey about the origins and
meanings of the concept of collective memory in the early work by Halbwachs
(1925;\ 19xx) and highlights the double meaning of the term: `collected
memory' as some kind of aggregate of individual memories, and a
`collectivist' as compared to individualistic concept of something that the
members of a group such as a nation have in common and that is an important
tool to generate a common identity. If collective memory is just the
aggregated sum of memories of a given group, the mechanisms of self-serving
bias and transformation of memory through activation might be at work much
like for individual memory.}

Diverging views and memories, if they exist, might not only be an outcome of
such individual processes of self-serving memory biases. Biased public news
and newspaper reports that differ between countries could play an important
role. If the news reporting during the crisis and in the years that followed
had national biases that also went along the dividing line between borrower
countries an lender countries, this might contribute to a divergence in
memories. In combination with the discussed plasticity of memory in the
phase of reactivation, divergent media news might establish a channel that
strengthens the self-serving bias:\ such media news activate memories, and
might transform the memory before it is stored again.

The causal relationship between effects on the individual citizen level and
the media need not be unidirectional. It might interact and compound: common
institutions inside a nation, such as common exposure to the same public
media and other public institutions might intensify information exchange
inside the group, might cause a continuous transformation of this aggregate
and might even strengthen and homogenize the national collections of
memories. For discussions see Rigney (2018) and Roediger and Abel (2015;\
361) who conclude: "Such collective memories probably boost group identity
and shape social and political discourse. In particular, studies of how
various groups remember `the same' events so differently may help to uncover
important psychological factors at work in group dynamics and conflict."

As we have data from 2009/2010 (experts/general population), we can address
the question whether the borrower or lender position are relevant for
differences in how the crisis is remembered and assessed. As we do not have
panel data or time variation, we will not be able to assess the dynamics of
differences in collective memories.
