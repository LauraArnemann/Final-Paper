\section{Introduction}

We examine collective memories of the European debt crisis that
began in 2010: memories of experts and non-experts, considering whether they are from program countries or non-program countries.
The European debt crisis partitioned the countries inside the Eurozone. Some
Eurozone member countries experienced fiscal and financial conditions that
dramatically worsened and were in danger of losing access to the capital
market for refinancing their public loans. The other countries had to
fear contagion effects and spillovers of possible national
bankruptcies in the more severely affected countries.\ Five countries
(Greece, Ireland, Portugal, Spain and Cyprus) applied for, and signed a
memorandum of understanding with the member countries of the European
Monetary Union. They received financial aid or
guarantees and accepted supervised mandatory structural reforms. We refer to the program
countries as the 'borrower countries'. Other countries used their fiscal credibility to provide these guaranties and requested
and participated in a monitoring of the process of structural reforms. We
refer to them as 'lender countries'.

Will citizens from borrower and lender countries have similar memories about who
applied for a memorandum of understanding? How do they remember
whether the lender or the borrower countries pushed for the memoranda of understanding and whether
the memoranda of understanding mainly benefited the lender countries or the
borrower countries? And if there are differences between borrower and lender countries, do these differ between economic experts and non-experts? We also
investigate how the programs were perceived in the countries
and how the crisis interventions influenced the relationship between European
countries.\ 

We consider the views of experts polled by CESifo's World Economic Survey (WES) panel of experts and non-experts in European countries in an internet questionnaire (https://prolific.co) to answer the same set of questions. The results do not suggest that the answers between experts from borrower countries and experts from lender countries diverge. However, differences emerge in how non-experts from the borrower and lender countries remember which countries signed a memorandum, why they signed it, and how they assess the measures taken to address the crisis. We employ data from 2018/2019 (experts/non-experts) and examine whether the borrower or lender position predicts how the crisis is remembered and assessed. Our data does not include variation over time. We therefore cannot examine how collective memories developed over time.

Memory formation has been primarily studied in psychology, neuroscience and sociology.
Insights about the plasticity of memory\footnote{%
The physiological basics for how memory is formed, kept and reactivated have
been explored. As Dudai and Edelson (2016;\ 276)\ describe, "When the memory
is retrieved, it seems to re-enter a transient phase in which it again
becomes susceptible to the same amnesic agents that were effective in the
original consolidation window (\cite{dudai}; \cite{nader}; Sara,
2000)." Brain sciences hence suggest that memory enters a state of
plasticity when it is reactivated and the copy that is then stored might
differ from the one that has been activated (see \cite{agren}, and \cite{lee}).}, and a tendency to memorize in a 'self-serving' way
(\cite{bell}) have been combined. If this self-serving bias
affects the process of reactivating, transforming and storing memory, it may well
cause memory with a self-serving bias.

A self-serving bias has been shown in memories of informal credit relationships (\cite{dezso}). Using data about informal credit
relationships between relatives and friends, \cite{dezso} find that borrowers and
lenders diverge in their memories about these credit relationships in self-serving
 ways.
Borrowers and lenders partially diverge in how they recall the credit event
as such, the conditions of the loan they agreed upon, and the mutual
interactions that made the contract come about. This divergence typically
has a self-serving bias. For instance, compared to the lenders, borrowers
tend to recall less if they exerted moral pressure to receive the loan and
are more likely to recall that the credit was generously offered to them.

The lenders and borrowers in the European sovereign debt crisis are
nations, no individuals, but the line of reasoning may well be
similar: when investigating the recollections of fiscal credit relationships or
loan guarantees between nations, the mental processes of the formation,
storage and reactivation of memory are still those of individuals.
The formation of memories of individuals who belong to the same group might
follow the same logic and physiological laws as in private
borrowing-lending relationships. Overall, individuals in borrower
countries and individuals in lender countries might recall differently
several aspects of the credit relationship similar to individuals in a private credit relationship. 

\\

Diverging views and memories might not only be an outcome of
such individual processes of self-serving memory biases but might be reinforced 
by biased public news and newspaper reports. The causal relationship between effects on the individual citizen level and
the media is ambiguous. Common
institutions inside a nation, such as common exposure to the same public
media and other public institutions might intensify information exchange
inside the group, might cause a continuous transformation of this aggregate
and might even strengthen and homogenize the national collections of
memories. For discussions see \cite{rigney} and Roediger and Abel (2015;\
361) who conclude: "Such collective memories probably boost group identity
and shape social and political discourse. In particular, studies of how
various groups remember `the same' events so differently may help to uncover
important psychological factors in group dynamics and conflict." \cite{baumeister} describe that groups employ techniques such as selective omission, fabrication of alternative narratives and exaggeration or embellishments of events to let their group appear in a favorable light. \cite{abel} show self-serving group bias in country's assessment of the Second World War. Citizens overestimate their country's contribution to the Second World War. 

We relate to studies which investigate differences between the opinion of economic experts and non-experts. \cite{johnston} find that citizens are influenced by the opinion of economic experts, more so if the issue at stake is highly technical and less ideological. However, the issue at stake does not appear to be the only determinant of divergence in the opinion of experts and non-experts. In a survey of economic experts and a representative sample of the U.S. population \cite{sapienza} find that the divergences in opinions is increasing in the level of agreement among economic experts. The WES has also been employed as a tool to measure discrepancies in the opinion of experts and non-experts. \cite{roth} find that experts and non-experts assess the impact of macroeconomic shocks quite differently.

Our study is also related to studies in political science.
%We conduct an inter-country comparison  of memories of the events surrounding the European debt crisis and compare the memories of economic experts to regular citizens. 
\cite{bechtel} examine how the political orientation of German citizens influences attitudes towards the European debt crisis. The results show that traditional left/right cleavages do not explain participants perception of the European debt crisis but rather moderateness of political views. More cosmopolitan survey participants are less opposed to the European rescue program. This finding is corroborated by \cite{kuhn} who use survey data: ``cosmopolitans" have a lower propensity to discriminate between national recipients of transfers and international recipients of transfers than non-cosmopolitans. We designed a survey specific to the European debt crisis and ask experts and non-experts in European Union member states. We compare the memories of economic experts and non-experts. Examining a nation-serving bias in memorizing the European debt crisis is new. 


