\documentclass[12pt]{article}
\usepackage{array}
\usepackage{graphicx} % Allows including images
\usepackage{booktabs}
\usepackage{tabularx}
\usepackage{siunitx}
\usepackage{ragged2e} 
\usepackage{amssymb,amsmath,amsfonts,booktabs,eurosym,geometry,ulem,graphicx,caption,color,setspace,sectsty,comment,footmisc,caption,natbib,pdflscape,subfigure,array,hyperref, tabularx, siunitx, ragged2e}
\usepackage{float}
\usepackage{subcaption}
\usepackage{authblk}
\normalem

\onehalfspacing
\newtheorem{theorem}{Theorem}
\newtheorem{corollary}[theorem]{Corollary}
\newtheorem{proposition}{Proposition}
\newenvironment{proof}[1][Proof]{\noindent\textbf{#1.} }{\ \rule{0.5em}{0.5em}}

\newtheorem{hyp}{Hypothesis}
\newtheorem{subhyp}{Hypothesis}[hyp]
\renewcommand{\thesubhyp}{\thehyp\alph{subhyp}}

\newcommand{\red}[1]{{\color{red} #1}}
\newcommand{\blue}[1]{{\color{blue} #1}}

\newcolumntype{L}[1]{>{\raggedright\let\newline\\arraybackslash\hspace{0pt}}m{#1}}
\newcolumntype{C}[1]{>{\centering\let\newline\\arraybackslash\hspace{0pt}}m{#1}}
\newcolumntype{R}[1]{>{\raggedleft\let\newline\\arraybackslash\hspace{0pt}}m{#1}}

\geometry{left=1.0in,right=1.0in,top=1.0in,bottom=1.0in}

\begin{document}
\title{Divergence of Collective Memory on European Financial Entrenchment}
\author{Laura Arnemann\textsuperscript{}\thanks{\noindent\textsuperscript{}University of Mannheim}\ , Kai A. Konrad\textsuperscript{}\thanks{\noindent\textsuperscript{}Max Planck Institute for Tax Law and Public Finance} \ and Niklas Potrafke \textsuperscript{}\thanks{\noindent\textsuperscript{}LMU Munich and ifo Institute; potrafke@ifo.de}}
\maketitle

\begin{abstract}
We examine the extent to which collective memory on the 2010 European debt crisis differs between citizens from borrower and lender countries. We employ new survey data for economic experts and non-experts. The results do not suggest that experts from borrower countries have different views than experts from lender countries. Economic expertise may well help to remember issues of the European debt crisis and to prevent nation-serving bias. By contrast, non-experts from borrower countries have different memories about the crisis than non-experts from lender countries. There is evidence for a nation-serving bias. Our results suggest that public debt relationships between nations influence how citizens of the individual countries perceive the contracting countries.
\end{abstract}
\clearpage
\section{Introduction}

This paper studies collective memories of the European debt crisis that
became salient in 2010. We explore differences in collective
memories between different groups:\ memories of citizens and experts, sorted
according to whether they are in program countries or non-program countries.
The European debt crisis partitioned the countries inside the Eurozone. Some
Eurozone member states experienced fiscal and financial conditions that
dramatically worsened and were in danger of losing access to the capital
market for refinancing their public loans. The remaining countries had to
fear contagion effects and other negative spillovers of possible national
bankruptcies in the more severely affected countries.\ Five countries
(Greece, Ireland, Portugal, Spain and Cyprus) applied for, and signed a
memorandum of understanding with the member countries of the European
Monetary Union. As part of these agreements, they received financial aid or
guarantees, but also accepted supervised mandatory structural reforms.
Simplifying and abbreviating these measures taken, we refer to the program
countries as the 'borrower countries'. On the other side were the countries
that used their fiscal credibility to provide these guaranties and requested
and participated in a monitoring of the process of structural reforms. We
refer to them as 'lender countries'.

The key question we ask is whether there are differences in how
citizens and experts from borrower and lender countries remember the crisis.
Will citizens from the two country groups have similar memories about who
actually applied for a memorandum of understanding? How do they remember
whether the lender or the borrower countries pushed more for it and whether
the memoranda of understanding mainly benefited the lender countries or the
borrower countries? And if there are biases along the sets of country
groups, do these differ between regular citizens and experts? We also
investigate how these programs were perceived in the respective countries
and how these crisis interventions affected the relationship between European
countries more generally.\ 

To study these questions we first considered possible divergence in the
views of experts. For this purpose we included specific questions to the
World Economic Survey (WES) panel of experts. Then we paid a larger number
of members of the general population in countries in Europe in an internet
questionnaire (https://prolific.co) to answer the same set of questions. We
find no systematic divergence in the answers between the WES experts from
borrower countries and experts from lender countries. However, significant
differences emerge in how the citizens from the two groups of countries
remember what countries signed a memorandum, why they signed it, and how
they assess the measures taken to address the crisis.

The empirical analysis of divergence of memories along the lines of borrower
countries and lender countries is motivated by several results of previous studies.
Memory formation has been studied in psychology, neuroscience and sociology.
Insights about the plasticity of memory\footnote{%
The physiological basics for how memory is formed, kept and reactivated have
been explored. As Dudai and Edelson (2016;\ 276)\ describe, "When the memory
is retrieved, it seems to re-enter a transient phase in which it again
becomes susceptible to the same amnesic agents that were effective in the
original consolidation window (Dudai, 2012; Nader et al., 2000; Sara,
2000)." Brain sciences hence suggest that memory enters a state of
plasticity when it is reactivated and the copy that is then stored might
differ from the one that has been activated (see Agren 2014, and Lee, Nader
and Schiller 2017).}, and a tendency to memorize in a 'self-serving' way
(see Bell and Echterhoff 2014) have been combined. If this self-serving bias
affects the process of reactivating, transforming and storing memory, it can
cause memory with a self-serving drift.

This combination of plasticity of memory and the self-serving drift has been
applied by Desz\"{o} and Loewenstein (2012) to hypothesize self-serving
memories on informal credit relationships. Using data about informal credit
relationships between relatives and friends, they found that borrowers and
lenders diverge in their memories about these credit relationships in
 ways that appear to be self-serving.\footnote{%
Borrowers and lenders partially diverge in how they recall the credit event
as such, the conditions of the loan they agreed upon, and the mutual
interactions that made the contract come about. This divergence typically
has a self-serving bias. For instance, compared to the lenders, borrowers
tend to recall less if they exerted moral pressure to receive the loan and
are more likely to recall that the credit was generously offered to them.}

The lenders and borrowers in the European souvereign debt crisis are
nations, rather than individuals, but the line of reasoning could be
similar: when analyzing the recollections of fiscal credit relationships or
loan guarantees between nations, the mental processes of the formation,
storage and reactivation of memory are still those of single individuals.
They formation of memories of individuals who belong to the same group might
follow the same logic and pyhsiological laws as in the context of private
borrowing-lending relationships. In the aggregate, individuals in borrower
countries and individuals in lender countries might recall differently which
side was the driving force for international credit relationships: the
lender-side interpretation might more often be that the borrower country was
desperately seeking help, whereas the borrower side might tend to identify
reasons why the lender-side forced the program upon the borrower country.
The two groups might also differ in their assessments about who eventually
benefited more from the conditional financial help program.\footnote{%
A large literature discusses collective memory, in difference to the memory
of individuals. Olick (1999) provides a lucid survey about the origins and
meanings of the concept of collective memory in the early work by Halbwachs
(1925;\ 19xx) and highlights the double meaning of the term: `collected
memory' as some kind of aggregate of individual memories, and a
`collectivist' as compared to individualistic concept of something that the
members of a group such as a nation have in common and that is an important
tool to generate a common identity. If collective memory is just the
aggregated sum of memories of a given group, the mechanisms of self-serving
bias and transformation of memory through activation might be at work much
like for individual memory.}

Diverging views and memories, if they exist, might not only be an outcome of
such individual processes of self-serving memory biases. Biased public news
and newspaper reports that differ between countries could play an important
role. If the news reporting during the crisis and in the years that followed
had national biases that also went along the dividing line between borrower
countries an lender countries, this might contribute to a divergence in
memories. In combination with the discussed plasticity of memory in the
phase of reactivation, divergent media news might establish a channel that
strengthens the self-serving bias:\ such media news activate memories, and
might transform the memory before it is stored again.

The causal relationship between effects on the individual citizen level and
the media need not be unidirectional. It might interact and compound: common
institutions inside a nation, such as common exposure to the same public
media and other public institutions might intensify information exchange
inside the group, might cause a continuous transformation of this aggregate
and might even strengthen and homogenize the national collections of
memories. For discussions see Rigney (2018) and Roediger and Abel (2015;\
361) who conclude: "Such collective memories probably boost group identity
and shape social and political discourse. In particular, studies of how
various groups remember `the same' events so differently may help to uncover
important psychological factors at work in group dynamics and conflict."

As we have data from 2009/2010 (experts/general population), we can address
the question whether the borrower or lender position are relevant for
differences in how the crisis is remembered and assessed. As we do not have
panel data or time variation, we cannot assess the dynamics of
differences in collective memories.


\section{The surveys }
In August 2018, we asked economic experts from the World
Economic Survey (WES) on their views about the financial
entrenchment following the European debt crisis. The WES was a quarterly
survey conducted by the ifo Institute (it was stopped in 2020). The WES sample includes 517
participants from EU member countries, among them 90 experts from program countries.
The survey regularly includes questions about overall economic development
from European and non-European experts such as economic growth and
inflation. The WES in August 2018 included our
questions on the European debt crisis (see the Appendix for the
questionnaire). \footnote{Scholars use the WES to include special modules. See, for example, \cite{mosler}.}

We asked non-experts by using the website
prolific.co. In contrast to other crowdsourcing platforms such as Mturk,
Prolific is a platform specifically designed to recruit participants for
academic research.\footnote{\cite{Peer} show that participants from
the platform prolific perform better than participants from other
crowdsourcing platforms.} In exchange for their participation in surveys or
experiments, participants receive a financial reward. The survey was
distributed to 1702 participants in August 2019, 498 of these participants
came from program countries. To ensure that our participants had an
opportunity to actively remember the events during the European debt crisis
we restrict our sample to include only participants older than 25.

The same questions were used in both surveys, and in
unchanged ordering. The appendix shows the country composition of survey
participants for both samples.\footnote{%
The WES sample and the prolific sample differ. The differences are based on the expert status of members
of the WES sample. The majority of participants from the prolific sample,
around 65 percent are younger than 35, whereas the majority of participants
from the WES sample are between 35 and 55. Participants from the WES sample
also have a higher level of education than participants from the
prolific sample, 60 percent of participants hold a PhD. Nonetheless the
majority of participants from the prolific sample have completed tertiary
education. More than 80 percent of WES participants are male, whereas in the
prolific sample there is an equal share of men and women.} We refer to the
WES sample as the expert sample and to the prolific sample as the non-expert
sample. 

\section{Questions, hypotheses and findings}

The questionnaire starts with recalling the European debt
crisis that began in 2010, and the \textit{Memoranda of Understanding}
that Greece, Spain, Portugal, Ireland and Cyprus signed with the European
Union, the European Central Bank and the International Monetary Fund that  
established financial aid combined with economic adjustment. We refer to
these contracts as \textit{aid\&reform} programs. 

We asked the respondents for each of the 19 EMU member whether
this was a program country, where a respondent could answer "yes",
"no", or "I don't know". The question has two purposes. First, it provides
information about the aggregate state of information. There are multiple
ways to aggregate the answers from 19 questions. %\footnote{%
%A very small number of experts found this question imprecisely stated. \emph{%
%We omit these replies the statistical analysis of this question?}} 
We
calculate a simple score:\ we count the number of program countries
correctly identified and deduct the number of wrong answers (both if a
respondent names a non-program country as program country or if the
respondent thinks that a program country did not sign a memorandum). Thus, the maximum score participants can obtain is 5, the minimum score -19. On
average, these numbers are

\begin{equation*}
\begin{tabular}{lll}
& Experts & Non-Experts \\ 
from borrower countries & $%
\begin{array}{c}
2.48 \\ 
(2.62)%
\end{array}%
$ & $%
\begin{array}{c}
1.10 \\ 
(2.74)%
\end{array}%
$ \\ 
from lender countries & $%
\begin{array}{c}
.78 \\ 
(3.50)%
\end{array}%
$ & $%
\begin{array}{c}
-.96 \\ 
(3.59)%
\end{array}%
$%
\end{tabular}%
\end{equation*}%
Experts were somewhat better informed than non-experts and participants from borrowers countries were somewhat better informed than participants from lender countries. The scores for experts from borrower and lender countries (2.48 and 0.78) are larger than the scores for non-experts from borrower and lender countries (1.10 and -0.96). The differences in the individual scores do, however, not turn out to be statistically significant.

We now turn to the questions that examine collective
self-serving memory biases. We ask whether country
origin matters for respondents' opinions about the \textit{reasons} for why lender countries wanted to engage in the credit relationship. 
\begin{figure}[h!]
\caption{Reasons of the lender countries for entering the rescue program}
    \begin{center}
    \includegraphics[scale=1.2]{graph2.pdf}
    \label{fig:Figure1}
    \end{center}
     \tiny
     \begin{tablenotes} 
    {The exact wording of the question is: In your opinion, what is the main reason why these countries entered these programmes 2a The lender countries wanted to help the borrower countries; 2b The lender countries wanted to help themselves to avoid a major crisis at home; 2c The lender countries wanted to force their desire for institutional change upon the borrower countries. Participants could choose the options strongly agree, slightly agree, slightly disagree and I don't know. We exclude all participants who answered with I don't know.\\
    The whiskers represent the 95 \% confidence intervals}
    \end{tablenotes}
\end{figure}

The nation-serving hypothesis is:\ respondents from borrower countries tend
to agree less frequently than respondents from lender countries 
that the lender countries wanted to help. They agree more that
the lender countries wanted to help themselves, and they agree more 
often that the lender countries wanted to force institutional change on the
crisis countries.\textit{\ }

The nation-serving hypothesis is not rejected based on our data (\autoref{fig:Figure1}). The share of non-experts from lender countries agreeing that lender countries wanted to help the borrower countries (62\%) is larger than the share of non-experts from borrower countries (48\%)
The difference in assessments between
non-experts from borrower and lender countries are large and statistically
significant at the 1 \% level for the aspect of reforms. There is also some disagreement in the expert sample. The differences, however, do not turn out to be statistically significant. \\
It is conceivable that
many non-expert respondents in the borrower countries were suspicious that
the lender countries had an institutional-reform agenda that goes beyond the
idea of helping each other. Participants from borrower and lender countries largely agree that lender countries wanted to avoid a crisis at home both in the expert and non-expert sample. The nation-serving bias manifests itself again in the non-expert sample in the question about the desire for institutional change among the lender countries. 62 \% of non-experts from lender countries agree with this statement compared to 72 \% among the lender countries. Again, we can observe differences in answers among the expert sample as well which are again not statistically significant. 
It is conceivable that experts, apart from simply being
better informed, often identify less with their own countries of origin and have 
a more cosmopolitan orientation. (At
least, this was the hypothesis that guided us to subject non-expert
nationals from the different countries to the same survey). Therefore, the
forces for developing a nation-bias might be less strong for experts than for non-experts. 

\begin{figure}
    \begin{center}
      \caption{Driving Force and Beneficiaries}
    \includegraphics[scale=1.2]{graph3.pdf}
  
    \label{fig:figure2}
    \end{center}
    \tiny
    \begin{tablenotes} 
    {The exact wording of the questions is: 3. In formal terms, the borrower countries that signed a memorandum had to apply for support. But thinking about the true motivations and the political processes behind these events, which of the following three alternatives corresponds most closely to your perceptions. Answer options: The borrower countries wanted it, the lender countries were more reluctant; The lender countries wanted it, the borrower countries were more reluctant; Both wanted it equally; I don't know \\
   Question 4: Who do you think mainly benefited from the rescue program. Answer options: The borrower countries; The lender countries; Both groups of countries benefited equally; I don't know. We exclude all participants who answered with I don't know. \\
   The whiskers represent the 95 \% confidence intervals.}  
    \end{tablenotes}
\end{figure}
We
turn now to the next two questions.
Question 3 is similar to question 2 and asks about underlying motivations
and intentions. Question 4 asks about the beneficiaries of the aid\&reform
programs. The questions corroborate that the difference in nation-bias is a more systematic pattern (\autoref{fig:figure3}). Both questions show very similar patterns for non-expert
respondents, depending on whether they are from borrower countries or from
lender countries. The results suggest large and significant nation-biases. Non-experts
from borrower countries tend to see the lender countries as the main
motivators for these programs, in any case, more than respondents from the
borrower countries. This result corroborates the results on
self-serving memories from private informal lending as in \cite{dezso}, who report a tendency to assign the initiative\
for such contracts to the respective other side of the credit relationship. 

Non-experts from borrower and lending countries also differ in
their perceptions of who mainly benefited from the aid\&reform programs. The
bias tends to assign the benefits more to the counterparty group:\
respondents from borrower countries think more often than respondents from
lender countries that the lender countries are the major beneficiaries. These numbers
suggest that there is a nation-serving bias. 

The responses of experts do not seem to be prone to a country-group
bias. Whether an expert is originally from a borrower country or a lender
country does not affect their assessments about the
motivations driving the aid\&reform programs, nor their assessments about
who was the main beneficiary. \footnote{98 percent of experts also works in their country of origin }  Interestingly, there is a nation-disfavoring bias in the sample of experts. Experts from lender countries are 12 percentage points more likely to agree that lender countries were the driving force and the main beneficiaries of the rescue program. However, these differences do not turn out to be statistically significant.

We now turn to questions addressing how the groups of respondents
assess the implications of the aid\&reform programs for feelings among the
populations in the borrower and the lender
countries. The respondents' perceptions might be formed by direct
observations in the countries or media reports, but their views about the
reasons and motivations for the aid\&reform programs and their views about
who actually benefited from these programs should correlate with their
assessments, and might cause their beliefs about these feelings. 

We ask whether respondents think that the rescue experience might
have caused feelings of \textit{guilt}, feelings of \textit{being
exploited}, and/or feelings of \textit{inferiority}. The answers to these
questions, and possible differences for respondents from lender countries or
from borrower countries hardly report
nation-serving collective memory biases, but might rather be the outcomes of
such biases. Consider a citizen in the borrower country, say Greece. This
citizen may truthfully state what he or she believes what the
co-citizens in Greece feel. For instance, whether or not they feel guilt.
This feeling might, among other factors, be a result of a (potentially
nation-serving) view about how the crisis came about, and how Greece
addressed the crisis. If the views about reasons/motivations and the
distribution of benefits for the aid\&reform programs are self-serving, then
the Greek citizen might not see any reason for guilt:\ the crisis was just
bad luck, Greece was forced to take up more than the right amount of
(painful) reforms, and the benefits from these measures went elsewhere.
Hence, the Greek population might not see any reason to feel guilty.
Assessing the positions of respondents in the lender countries, these might
think that the crisis was caused by poor pre-crisis politics in Greece, that
Greece got a lot of assistance (see the previous questions) and resisted to
follow the Troika advice on necessary reforms for recovery. What does this
perception mean for beliefs about guilt? The respondent from the lender
country might think:\ if I were them, for what they have done, I would feel
guilty. The respondent might also have second-order beliefs.\ The respondents
might understand that the citizens in Greece perceived matters in a
nation-serving way and therefore will not feel guilty. While this modifies
possible expected outcomes, overall this reasoning points at a
borrower-country bias towards not agreeing that the citizens of these
countries feel guilty. 

NIKLAS: WOLLEN WIR DIE FN 6 rausnehmen?
\footnote{Our findings can also be interpreted in an alternative way.  Having a self-serving or nation-serving bias might make individuals oblivious to
the way policies are received in other countries. 
\cite{dezso} refer to this phenomenon as having a ``blind spot" regarding the other party's 
feelings and emotions. The hypothesis on the existence of such a ``blind spot" is confirmed in our findings.
Citizens from lender countries are more likely to agree that they felt guilty, exploited and/or inferior as a 
consequence of the rescue program. The largest difference between lender and borrower countries occurs with regards to feeling exploited.
78 percent  of citizens from borrower countries state that they felt exploited due to the rescue program while only 61 percent of lender countries
agree to this statement. 
}

%Similar reasons might let us suspect that the respondents from the lender
%countries are more inclined to think their population feels exploited, or
%feel inferior (in the sense of humiliated). Figure xxx shows the answers.
%First statistics negate the existence of a blind spot among borrowers about the feelings
%from citizens to lender countries. 

\begin{figure}
\begin{center}
    \caption{Emotions of the borrower countries}
    \includegraphics[scale=1.2]{graph5_1.pdf}
    \label{fig:figure3}
    \end{center}
    \tiny 
      \begin{tablenotes} 
      {The exact wording of the question was the following: Please give assessments of the following questions: 5a) The rescue experience made many citizens in the borrower countries feel guilty 
      5b) The rescue experience made many citizens in the borrower countries feel exploited 5c) The rescue experience mad many citizens in the borrower countries feel inferior
      Answer options: strongly agree, slightly agree, slightly disagree, strongly disagree, I don't know. We exclude all participants who answered with I don't know. \\
      The whiskers represent the 95 \% confidence intervals.}
      \end{tablenotes}
\end{figure}

We might expect that experts have a view on these questions that is hardly influenced by their countries of origin, such that we would not find much
difference in the responses of participants who originated in the borrower
countries and those of participants who originated in the lender countries (\autoref{fig:figure3}).


Similar questions assess the perceptions in the groups about the
feelings in the lender countries. In particular, we asked the participants whether they
agree on that the aid\&reform programs made the citizens in the
lender countries feel exploited and disappointed (\autoref{fig:figure4}). The non-experts in the lender
countries should be prepared to report their own feelings. However, the
perceptions about these lender-country feelings by the respondents from the
borrower countries might be distorted for several reasons. If they believe
that the lender countries are the true winners of these programs and that
this is how the citizens in these countries feel about it, they should agree
less to the ideas that citizens in lender countries feel exploited and
disappointed. If, however, they have second-order beliefs and can
successfully place themselves in the shoes of lender-country citizens, they
might understand that their beliefs about how the crisis came about and who
benefited from the programs are quite different, and they might correctly
assess their true feelings of being exploited and disappointed. Overall,
however, we might expect that the lender-country respondents agree more
frequently than the borrower-country respondents to the possibly negative
feelings in lender countries:

\begin{figure}[h!]
  \begin{center}
       \caption{Emotions of the lender countries}
    \includegraphics[scale=1.2]{graph5_2.pdf}
 
    \label{fig:figure4}
    \end{center}
    \tiny
    \begin{tablenotes}
     {The exact wording of the question is: 5d) The rescue experience made many citizens in the lender countries feel exploited 5e) The rescue experience made many citizens in the lender countries feel disappointed
    Answer options: strongly agree, slightly agree, slightly disagree, strongly disagree, I don't know. We exclude all participants who answered with I don't know. \\
    The whiskers represent the 95 \% confidence intervals}
    \end{tablenotes}
\end{figure}

We compare non-experts and experts. The results do not suggest that respondents from borrower countries have different views than respondents from lender countries. The findings indicate that the respondents
in both groups of countries believe that the \textit{aid\&reform programs} triggered
bad feelings in both groups of countries. A large share of the respondents
thinks that citizens in borrower countries feel guilty, exploited, and
inferior, and a large share of respondents also thinks that citizens in
lender countries feel disappointed and exploited as well. 

We have also asked whether the \texti{aid\&reform programs} strengthened friendship between the citizens in the Eurozone. The theory of groups and conflicts shows that, when groups jointly master a major task which none of them could have mastered in isolation, they overcome negative attitudes and mutual spite between groups. The European debt crisis had the potential to be 
such a task. It might have strengthened friendship between these groups.
However, if both groups have nation-serving diverging biases about what
motivated the crisis and the solution method adopted, and about the
distribution of the benefits and costs of the method adopted, then we might
expect that a large fraction of the respondents of both groups think that
these programs did not strengthen friendship ties. 
\begin{figure}
\begin{center}
\caption{Impact on friendships}
\includegraphics[scale=0.5]{graph5_3.pdf}
\label{fig:figure5}
\end{center}
\tiny 
\begin{tablenotes}
  {The exact wording of the question is the following: 5f) The rescue experience strengthened friendships
   Answer options: strongly agree, slightly agree, slightly disagree, strongly disagree, I don't know. We exclude all participants who answered with I don't know. \\
     The whiskers represent the 95 \% confidence intervals}
    \end{tablenotes}
\end{figure}

The majority of participants from the expert and non-expert sample disagrees that 
the rescue package strengthened friendships (\autoref{fig:figure5}).\footnote{Due to a survey error this question was displayed as "The rescue experience strengthened friendship ties between borrower". The fraction of experts who answered this question with "I don't know" lies around 20 percent. This is very much in line with the frequency of "I don't know" responses throughout the survey. Hence, it seems plausible that participants correctly understood the question. } This holds for participants from 
both the expert and the non-expert sample. This finding is also in line with the nation-serving biases
we find throughout the survey. In light of the absence of such a bias among the expert sample 
it seems interesting that the level of agreement in this sample is also quite low. 
This might suggest that although experts might not have and be aware of a nation-serving bias
among citizens from the lender countries they are indeed aware of the consequences of such 
a nation-serving bias. 
\\

The answers to the individual questions are likely to
be interdependent. For example, individuals in lender countries who
think that the borrower countries pushed for the aid\&reform programs
(questions 2 and 3) may also think that these are the main beneficiaries
(question 3), and therefore think that the lender countries feel exploited
and disappointed. The direction of causality for respondents from the lender
countries would be%
\begin{equation*}
\begin{array}{ccccc}
\begin{array}{c}
\text{borrower countries} \\ 
\text{pushed for aid\&reform}%
\end{array}
& \rightarrow  & 
\begin{array}{c}
\text{borrower countries} \\ 
\text{are the main} \\ 
\text{beneficiaries of} \\ 
\text{aid\&reform}%
\end{array}
& \rightarrow  & 
\begin{array}{c}
\text{lender countries} \\ 
\text{feel exploited} \\ 
\text{and disappointed.}%
\end{array}%
\end{array}%
\end{equation*}


We evaluate this chain of causality empirically: 
\begin{table}[h!]
    \centering
    \begin{tabular}{c|c}
         &  \\
         & 
    \end{tabular}
    \caption{Empirical Assessment of chain of causation }
    \label{tab:my_label}
\end{table}

\begin{table}[h!]
\begin{center}

\begin{tabular}{l*{1}{cccc}}
\hline\hline
                    &\multicolumn{4}{c}{}                               \\
 The lender countries were the driving force                   &   4 & 5.1 & 5.2 &  5.3 \\
\hline
Answers of people disagreeing                &        0.32&        0.40&        0.63&        0.65\\
Answers of people agreeing                &        0.70&        0.40&        0.78&        0.72\\
\hline
Total               &        0.51&        0.40&        0.70&        0.68\\
\hline\hline
\end{tabular}
\end{center}
\begin{tablenotes}
\item \tiny Conditional on participants agreeing or disagreeing with the lender countries being the driving force for the aid & reform program we present the likelihood to state lender countries as an answer to question 4 (Who mainly benefited from the reform), Question 5.1 (The rescue experience made citizens in the borrower countries feel exploited); Question 5.2 (The rescue experience made many citizens in the borrower countries feel exploited) Question 5.3 (The rescue experience made many citizens in the borrower countries feel inferior)
\end{tablenotes}
\end{table}

\begin{table}[h!]
   \begin{center}
\begin{tabular}{l*{1}{ccc}}
\hline\hline
                    &\multicolumn{3}{c}{}                  \\
Lender countries were main beneficiaries                    &  5.1 &  5.2 &  5.3 \\
\hline
Answers of people disagreeing           &        0.35&        0.58&        0.58\\
Answers of people agreeing                   &        0.45&        0.81&        0.74\\
\hline
Total               &        0.40&        0.70&        0.66\\
\hline\hline
\end{tabular}
\end{center} 
\begin{tablenotes}
\item \tiny
Conditional on participants stating that lender countries benefiting from the aid & reform program we present the likelihood to agree with Question 5.1 (The rescue experience made citizens in the borrower countries feel exploited); Question 5.2 (The rescue experience made many citizens in the borrower countries feel exploited) Question 5.3 (The rescue experience made many citizens in the borrower countries feel inferior)
\end{tablenotes}
\end{table}
The results suggest that participants who agree with the statement that the borrower (lender) countries were the driving force behind the reform are also more likely to agree that the borrower (lender) countries were the main beneficiaries of the aid& reform package. The assessment of which party was the driving force does not influence participants opinions on the feelings that were evoked among borrower or lender countries. However, participants who agree with the statement that the lender countries were the main beneficiaries from the rescue program also show a higher likelihood to agree that the borrower countries felt guilty/ exploited or inferior due to the rescue program. For participants who agree that the 
respondents from the borrower countries who think that the lender countries
pushed for the aid\&reform programs, wanted to impose institutional reforms
on them are probably the same who think that the lender countries are the
main beneficiaries, and also the same ones who think that the borrower
countries have been exploited and would not feel guilt, but feel humiliated
("inferiority"?). The direction of causality is%
\begin{equation*}
\begin{array}{ccccc}
\begin{array}{c}
\text{lender countries} \\ 
\text{pushed for aid\&reform} \\ 
\text{and wanted to impose} \\ 
\text{structural reforms}%
\end{array}
& \rightarrow  & 
\begin{array}{c}
\text{lender countries} \\ 
\text{are the main} \\ 
\text{beneficiaries of} \\ 
\text{aid\&reform}%
\end{array}
& \rightarrow  & 
\begin{array}{c}
\text{borrower countries} \\ 
\text{feel exploited}%
\end{array}%
\end{array}%
\end{equation*}
\begin{table}
  \begin{center}
\begin{tabular}{l*{1}{ccc}}
\hline\hline
                    &\multicolumn{3}{c}{}                  \\
                    & 4& 5.4 &  5.5 \\
\hline
0                   &        0.24&        0.59&        0.61\\
1                   &        0.47&        0.65&        0.63\\
Total               &        0.33&        0.61&        0.62\\
\hline\hline
\end{tabular}
\end{center}
\begin{tablenotes}
\item \tiny Conditional on participants agreeing or disagreeing with the borrower countries being the driving force for the aid & reform program we present the likelihood to state borrower countries for question 4 (Who was the main beneficiary from the rescue program?), or agree with question 5.4 (The rescue experience made citizens in the lender countries feel exploited); Question 5.5 (The rescue experience made many citizens in the lender countries feel disappointed)
\end{tablenotes}
\end{table}

\begin{table} 
  \begin{center}
\begin{tabular}{l*{1}{cc}}
\hline\hline
                    &\multicolumn{2}{c}{}     \\
                    &  5.4  5.5 \\
\hline
0                   &        0.59&        0.60\\
1                   &        0.65&        0.65\\
Total               &        0.61&        0.62\\
\hline\hline
\end{tabular}
\end{center} 
\begin{tablenotes}
\item \tiny Conditional on participants stating that borrower countries did or did not benefit from the aid & reform program we present the likelihood to agree with Question 5.4 (The rescue experience made citizens in the lender countries feel exploited); Question 5.5 (The rescue experience made many citizens in the lender countries feel disappointed)
\end{tablenotes}
\end{table}
\\


\clearpage
\section{Results}
We present the results of our empirical analysis for the sample of experts (the WES sample) and the sample of non-experts (the prolific sample).  We estimate the following model by binary logit. 
\begin{equation*}
    Y_{ijc}= \alpha_{j}+ \beta *D_{c} + \gamma*X_{i}
\end{equation*}
Where $Y_{ijc}$ denotes the response of individual $i$ from country $c$ to question $j$ we will control for individual characteristics $X_{i}$ such as age, level of education, gender and employment status (affiliation for the expert sample). The dummy variable $D_{c}$ indicates whether the respondents nationality is Greek, Cypriotic, Spanish, Irish or Portuguese with the coefficient $\beta$ measuring the divergence in the answers of participants from program and non-program countries.For questions in which participants were asked to state their level of agreement we estimate the effect of being from a program country on the likelihood to strongly or slightly agree. For questions in which participants are asked to name the responsible party, we estimate the effect of being from a program country on the likelihood to state "Lender countries". \\ \\
\textbf{Baseline Results} 
When asked about the intentions of lender countries to initiate the credit relationship our hypothesis is verified among our non-expert participants.
 In the non-expert sample citizens from program countries are 10.9 percentage points more likely to agree that lender countries wanted to impose institutional change upon the borrower countries. They are 13.1 percentage points less likely to agree that the lender countries wanted to help the borrowing countries. However, there is no significant effect on agreeing that lender countries merely wanted to avoid a crisis at home. In the expert sample there are no statistically significant differences in the assessments of experts from program and non-program countries. Contrasting our hypothesis,  experts from program countries are more likely to agree that the lender countries wanted to help the borrowing countries. This effect does not turn out to be statistically significant.  \\ 
 \begin{figure}[H]
\begin{center}
     \caption{Intentions of the lender countries}
    
     \includegraphics[scale=0.8]{Question2_base.pdf}
     \label{fig:my_label}
      \end{center}
      \tiny
     \tablenotes{The exact wording of the questions were the following.  Question 2.1: The lender countries wanted to help the borrowing countries Question 2.2: The lender countries wanted to help themselves avoid a crisis at home Question 2.3: The lender countries wanted to impose institutional change upon the borrower countries  }
\end{figure}
 We continue to ask the participants about the emotions the rescue program evoked among citizens from the borrower and lender countries. The hypothesis that lenders have a blind spot regarding the feelings of citizens from borrower countries is confirmed. Participants from program countries in the non-expert sample are 9.2 and 9.3 percentage points more likely to agree that the rescue experience made them feel guilty and inferior. Further, the probability to agree to "The rescue experience made many citizens in the borrower countries feel exploited" increases by 16.9 percentage points among citizens from program countries. All effects are statistically significant at the 1 percent level. Among the sample of economic experts the effect sizes are smaller and not significantly different from zero. \\
 Interestingly, we do not find that borrowers have a blind spot regarding the emotions the program evoked among lender countries. There are only small differences in the likelihood to agree with certain statements between citizens from program and non-program countries which are not significant. There is no difference in evaluating the effect of the rescue program on the friendship between citizens. The results are similar in the expert and non-expert sample. \\

 We further predicted that borrower countries will be more confident that they will be able to repay outstanding debt. 
 When asked whether Greece \footnote{We only ask about Greece, since Greece remains the only country that has not repaid it's debt} will be capable of fully paying back it's debt, citizens from program countries show a 13.5 percentage point higher likelihood to agree in the non-expert sample and a 12.2 percentage point higher likelihood in the expert sample. This effect is again significant at the 1 percent level. \\
 \begin{figure} [h!]
    \begin{center}
    \caption{Sentiments of the borrower countries}
    \includegraphics[scale=0.8]{Question5_1_base.pdf}
    \label{fig:my_label}
    \end{center}
    \tiny
    \tablenotes{The exact wording of the questions are the following. Question 5.1: The rescue experience made many citizens in the borrower countries feel guilty; Question 5.2: The rescue experience made many citizens in the borrower countries feel exploited; Question 5.3: The rescue experience made many citizens in the borrower countries feel inferior. }
\end{figure}
\begin{figure}[h!]
\begin{center}
\caption{Sentiments among lender country citizens}

    \includegraphics[scale=0.8]{Question5_2_base.pdf}
    \label{fig:my_label}
    \end{center}
    \tiny
    \tablenotes{The sign in parantheses denoted the predicted differential effect. Question 5.4: The rescue experience made many citizens in the lender countries feel exploited; Question 5.5 The rescue experience made many citizens in the lender countries feel disappointed Question 5.6: The rescue experience strengthened friendships between citizens Question 7: Greece will fully pay back it's debt}
\end{figure}
 
%  \begin{figure}
%\caption{Assessment of the emotions the program evoked among different parties}
%\centering
%\begin{minipage}{.5\textwidth}
% \includegraphics[scale=0.5]{Question5_1_base.pdf}
 %\end{minipage}%
 %\hfill
%\begin{minipage}{.5\textwidth}
%\includegraphics[scale=0.5]{Question5_2_base.pdf}
%\end{minipage}
%\end{figure}
 We proceed to evaluate who initiated and who benefited from the credit relationship. 
In the non-expert sample citizens are 13.2 percentage points more likely to state that the lender countries were the driving force behind signing the memorandum. Further, they are 28 percentage points more likely to state that the lender countries were the main beneficiaries of the program and 20.3 percentage points more likely to state that lender countries benefited from the loans to Greece. All effects are statistically significant at the one percent level. We do not find statistically significant effects in the expert sample. Interestingly, the mean difference between expert and non-experts is negative, contrary to the anticipated effect. Experts from program countries are for example 12.4 percentage points less likely to state that citizens from lender countries were the main beneficiary from the program.\\
\begin{figure}[h!] 
\begin{center}
     \caption{Who initiated and benefited from the rescue program}
     \includegraphics[scale=0.8]{Question3_4_base.pdf}
     \label{fig:my_label}
     \end{center}
     \tiny
     \tablenotes{The sign in parantheses denotes the predicted differential effect.Question 3: Who was the driving force behind signing the memorandum; Question 4: Who was the main beneficiary of the program; Question 7: Who primarily benefited from the loans to Greece}
\end{figure}

\begin{figure}[h!] 
\begin{center}
     \caption{Situation in Greece}
     \includegraphics[scale=0.8]{Question6_7_base.pdf}
     \label{fig:my_label}
     \end{center}
     \tiny
     \tablenotes{The exact wording of the question was the following.   Question 6: Who primarily benefited from the loans to Greece; Question 7: Greece will fully pay back it's debt}
\end{figure}
\clearpage
\textbf{Sample Splits and Heterogeneity Analysis}
Program and non-program countries vary along other dimensions than the program and non-program distinction. Program countries share several characteristics which are likely to influence the estimates. Countries which were affected by the European debt crisis are predominantly Southern European and are located at the periphery of the European Union (measured by distance to Brussels). Further, all these countries experienced a substantial increase in debt levels, high levels of unemployment, low GDP growth and belonged to the Eurozone. Thus we conduct sample splits along these margins to test whether these variables drive our results. \footnote{ We again estimate our model on the subsample of countries which are Southern European, belong to the Eurozone,are located at the periphery of Europe,defined as countries in which the distance of the capital to Brussels is above the median, experienced above median debt and unemployment growth and below median GDP growth during the years 2007 and 2012.} The overall findings of our analysis remain unchanged among the expert and non-expert sample.
\\
Experts and non-experts differ along various dimensions which might influence the magnitude of effect and explain the observed differential impact of belonging to a program country. Experts are typically older, have a higher level of education and might also be more mobile than non-experts. We estimate our model for the sample of experts older than 35, tertiary education and who report to be living in a different country than their country of birth. The results of the non-expert sample do not change for the sample splits along age and education, but do show some changes along the variable mobility. Hence, being exposed to a more international environment could be driving the observed differences in results between the expert and non-expert sample. Additionally, we also redefine the program variable according to beliefs of the survey participants. We now estimate the effect of believing to live in a borrower country on the outcomes of our survey. Interestingly, the effect of believing to live in a program country is smaller than the effect of actually living in a program country. In addition to the heterogeneity analysis we also conduct several robustness checks. We estimate the model by ordered and multinomial estimation, we cluster standard errors at the country level, control for multiple hypothesis testing and drop inattentive respondents from our non-expert sample. All these checks do not yield substantially different results than our baseline estimation. A detailed overview of the results of our heterogeneity analysis and robustness checks can be found in the appendix. 



%\section{Heterogeneity Analysis and Robustness Checks}
Our results show that experts and non-experts show differences in their assessment of the European debt crisis. This finding is in line with work by \cite{roth} who using the WES and a representative sample show that experts differ from non-experts in their beliefs about the impacts of macroeconomic shocks. In our heterogeneity analysis we investigate which difference drives the observed results. We also examine whether individual factors influence the magnitude of the observed divergence. 
\\



\textbf{Socioeconomic Characteristics}
 Experts and non-experts differ along various dimensions such as education, age and gender.
 According to \cite{baumeister} events experienced at a younger age might have a more defining impact. Since the European debt crisis was accompanied for example by a high level of youth unemployment it might seem plausible that the divergences in memory might be more pronounced among the younger generation. Thus, we estimate the effect of belonging to a program country among participants older than 35 among the sample of non-experts.  The effect of the program variable on the likelihood to agree that the lender countries wanted to impose institutional change or that the rescue experience made citizens in the borrower countries feel guilty is smaller and the significance level decreases to 5 percent. For the remaining questions the overall magnitude and significance level of the program effect stays the same. 
\\
The level of education of participants may well influence the estimates. Participants with a higher level of education might have different political attitudes or consume and access different types of media. Hence, we split the non-expert sample and estimate our model only for participants reporting to have completed tertiary education. We do not find any change in the magnitude and significance level of effects for this subsample. 
\\
One dimension along which experts and non-experts differ is their degree of mobility. Experts working in think tanks or research institutes might live or have studied abroad for some time. Due to this circumstance experts might identify less with their nation than non-experts and consequently will not have a strong nation-serving bias. Unfortunately we don't have information about whether participants in the non-expert have lived or studied abroad. However, we can identify if people reported to be living in a different country than their country of birth. This applies to 25 percent of the non-expert subsample. 20.72 percent of participants from program countries and 25.45 percent from non-program countries report to be living in a different country than their country of birth. Estimating our model for this subsample changes the results quite a bit. Divergence between citizens from program- and non-program countries remains in the assessment if lender countries wanted to help borrowing countries and if the rescue experience made the citizens in the borrower countries feel exploited. For the other questions the difference in answers between program- and non-program countries becomes smaller and even vanishes completely for some questions. Hence, it appears that a higher level of mobility might be causal for the differential effects between expert and non-expert sample.  \\

\\
\textbf{Knowledge and beliefs} 
Participants differ in their level of knowledge about the European debt crisis. Some people fail to correctly identify their country as a borrower or lender country. An overview of the fraction of participants who knew their country's status can be found in the appendix. However, knowledge about the status of one's country does not appear to influence the observed effects. The estimates of the baseline model on the subsample of non-experts who could correctly identify their country does not change in comparison to the estimates for the full sample. 

\\\\
We redefine the program variable according to the beliefs of the survey participants. We now estimate the divergence in answers between participants which believed to be the national of a  lender country and participants which believed to be the national of a borrower country. Replacing the program variable by beliefs about belonging to a program country yields different results than the baseline model. In comparison to participants who believe to be lenders, participants who believe to be borrowers do not agree less that lender countries wanted to help borrower countries. They also do not agree more that the rescue program made citizens in the borrower countries feel guilty or inferior and or less likely to state that the lender countries were the driving force behind signing the referendum. Interestingly, differences emerge in the agreement about the feelings of citizens in the lender countries. Participants who believe they live in borrower countries are less likely to agree that citizens in the lender countries felt exploited or disappointed. They are further more likely to believe that the rescue program strengthened friendships between citizens. \\

\\
Our heterogeneity analysis yields some interesting observations about potential drivers of the observed differences between expert and non-expert sample. The comparison of means suggests that the observed differences are not driven by the opinion of experts resembling the opinion of non-experts from either program or non-program countries. 
Our analysis suggests that the observed difference between experts and non-experts cannot be explained by differential effects across age or education levels. However, it appears that non-experts which are more mobile do not show a strong nation-serving bias in their assessments of the European debt crisis. Interestingly, being able to correctly identify one's country as a program or a lender country does not change the observed magnitude of results. The magnitude of effects does change however, when redefining the program variable according to beliefs of people. This suggests that collective memory might work on a more subconscious level regardless of the level of information participants have. 
\\
\\
We also conduct various robustness checks. \\

\textbf{Ordered and Multinomial Estimation}
Participants have multiple answer possibilities. Depending on the question participants are able to rank their level of agreement on a scale of 1 to 4 or choose between the options borrower countries, lender countries or both equally. To control if differences between participants from non-program and program countries also emerge when including all answer possibilities as regression outcomes we estimate multinomial and ordered logit models. For all questions in which participants were asked to rank their level of agreement we estimate an ordered logit model, for questions in which participants were asked to name the responsible party we estimate an multinomial logit model. The significance level in the non-expert sample remains at the same level for all but one question. For the question which party was the driving force behind the referendum the significance level changes from 0.01 percent to 0.05 percent. In the expert sample all results remain insignificant. For the question if Greece will repay it's debt the divergence in effects is only significant at the 5 percent level and no longer the 1 percent level. In the majority of cases the the magnitude of effects is the same for different answer possibilities. This suggests that differences between program and non-program countries exist across all answer possibilities. \\

\\
\textbf{Inattentive Respondents}
Since we distributed our survey for the non-expert sample online we will also control for the influence of inattentive respondents in the non-expert sample. When distributing the survey online we already included an attention check when asking participants about socioeconomic characteristics. All participants failing this attention check were excluded from the survey. Further, we exclude all participants at the top 10 $\%$ and bottom 10 $\%$ of the survey time distribution. Excluding these participants does not change the inferences of our baseline estimation in the non-expert sample.\\


\\
\textbf{Clustered Robust Standard Errors} 
We cluster standard errors on the country level. Since we only collected data from 24 countries we adjust for the small number of clusters using the wild bootstrap method for logit regressions as suggested by \cite{cameron}.\footnote{We use the Stata command developed by \cite{roodman}} Inferences change for some questions when applying this method. Question 2.c becomes insignificant, questions 5a and 5c loose some significance and become significant at the 5 respectively 10 percent level.  \\

\textbf{Multiple Hypothesis Testing}
We also control for multiple hypothesis testing by adjusting our p-values using the Bonferroni Method. We adjust p-values by the number of questions we ask our participants. The Bonferroni correction does not change the significance level of our results. 


%\section{Case Study: The situation in Greece}
We ask participants how they assess the situation in Greece.  We examine Greece in detail for many reasons. Among the borrower countries Greece received by far the 
highest volume of loans and was most severely affected by the European debt crisis. Hence, the crisis in Greece was by far the most salient and widely debated. 
The \textit{aid$&$reform} gave rise to protests in Greece. Also Greece is the only borrower country which has not yet repaid its loans. 
In question 6 we ask, similar to question 4, which party benefited most from loans
to Greece. Again, we expect a nation-serving bias between citizens from borrower
and lender countries. The resistance of the population in Greece was large and the consequences for the population were drastic. This might give rise to a higher level of agreement between borrower and lender countries alike that the lender countries
were the main beneficiaries of the rescue program to Greece causing the divergence in opinions between lender and borrower countries to 
disappear. However, it is also conceivable that there is some level of solidarity between other lender countries and Greece causing 
the divergences to remain largely constant. We ask whether Greece will repay outstanding debt. According to 
our nation-serving hypothesis we would expect citizens from Greece (and potentially other borrower countries) to have a higher degree of confidence 
regarding the repayment of the outstanding level of debt than respondents from lender countries. \autoref{fig:figure10}

\begin{figure}[h!]
    \caption{The situation in Greece}
\begin{center}
    \includegraphics[scale=1.2]{graph6.pdf}

    \label{fig:figure10}
    \end{center}
    \tiny
    \begin{tablenotes}
     {The exact wording of the questions is: The two remaining questions are specifically about Greece. Question 6: Greece will fully pay back it's debt ; Answer options strongly agree, slightly agree, slightly disagree, strongly disagree, I don't know
    Question 7: Who primarily benefited from the loans granted to Greece; Answer options:  Greece, The lender countries, Both benefited equally, I don't know. We exclude all participants that answered the question with I don't know. \\
    The whiskers represent the 95 \% confidence intervals. }
    \end{tablenotes} 
\end{figure}


The results show both non-experts and experts from borrower countries (0.33 and 0.30) believe to a larger extent than non-experts and experts from lender countries (0.18 and 0.16) that Greece will fully pay its debt; a result that indicates a nation-serving bias. The observed differences are statistically significant for the sample of non-experts and experts.  

 \clearpage
 The results also show that the fraction of non-experts from lender countries (0.35) is much smaller than the fraction of non-experts from borrower countries (0.56) believing that Greece benefited from the rescue program. Inferences based on the unconditional correlations in Figure 10 are corroborated by the conditional correlations reported in Figure 11.
 NIKLAS: WAS wollen wir mit diesem ERGEBNIS MACHEN? 
 \\
\begin{figure}[h!] 
\begin{center}
     \caption{Situation in Greece}
     \includegraphics[scale=0.8]{Question6_7_base.pdf}
     \label{fig:my_label}
     \end{center}
     \tiny
     \tablenotes{The exact wording of the question is.   Question 6: Who primarily benefited from the loans to Greece; Question 7: Greece will fully pay back it's debt}
\end{figure}
\section{Robustness Checks}

Borrower and lender countries vary along other dimensions than the borrower/ lender distinction. Program countries share several characteristics which are likely to influence the estimates. Countries which were affected by the European debt crisis are predominantly Southern European and are located at the periphery of the European Union (measured by distance to Brussels). Further, all these countries experienced a substantial increase in debt levels, high levels of unemployment, low GDP growth and belonged to the Eurozone. Thus we conduct sample splits along these margins to test whether these variables drive our results. \footnote{ We again estimate our model on the subsample of countries which are Southern European, belong to the Eurozone,are located at the periphery of Europe,defined as countries in which the distance of the capital to Brussels is above the median, experienced above median debt and unemployment growth and below median GDP growth during the years 2007 and 2012.} The overall findings of our analysis remain unchanged among the expert and non-expert sample.
\\
In addition to controlling for the influence of macroeconomic variables we also conduct several other robustness checks. We estimate the model by ordered and multinomial estimation, we cluster standard errors at the country level, control for multiple hypothesis testing and drop inattentive respondents from our non-expert sample. All these checks do not yield substantially different results than our baseline estimation. A detailed overview of the results of our heterogeneity analysis and robustness checks can be found in the appendix. 


\section{Conclusion} 
The 2010 European public debt crisis influenced policies, politics and voters' perceptions. For example, domestic policy-makers introduced measures such as fiscal rules to handle increasing public debt and budget deficits. The European Central Bank pursued expansionary monetary policies: it introduced the Outright Monetary Transactions (OMT) program and decreased interest rates to zero. Rescue programs for Cyprus, Greece, Ireland, Portugal and Spain were designed. 

We examined citizens' views about the European public debt crisis. In particular, we investigated collective memory of the European public debt crisis and disentangled views of citizens from borrower countries and citizens from lender countries. During the public debt crisis, media reports have suggested that citizens from borrower and lender countries have different views on the crisis and how to handle it. Studies in psychology suggest that individual borrowers and lenders remember credit relationships in different manners (\cite{dezso}): memories are influenced by a self-serving bias. It is conceivable that such self-serving bias gives rise to a nation-serving bias in the memory of national credit relationships. We examine a nation-serving bias in memories of national credit relationships. Doing so is new.

We compiled new data measuring collective memories on the European debt crisis. We asked economic experts by using CESifo's World Economic Survey and non-experts by using the provider prolific. The results suggest that experts from lender and borrower countries have quite similar views about the European public debt crisis. The views of non-experts are, by contrast, influenced by a nation-serving bias. The bias relates to memories on why countries signed a memorandum and how they assess the measures taken to address the crisis. These results corroborate empirical studies about citizens' misperceptions about macroeconomic policies and outcomes. Citizens evaluate macroeconomic policies and outcomes much better when their preferred political party is in office than when parties govern that they do not support (on partisan bias see, for example, \cite{evans}, \cite{gerber}, \cite{gillitzer}, \cite{bachmann}). 

We observe a much smaller nations-serving bias among non-experts when we examine non-experts that left their country of birth. Cosmopolitans are less likely to evaluate European policies with a nation-serving bias than non-Cosmopolitans (see also \cite{bechtel}).

We believe that citizens' misperceptions and nation-serving bias in memorizing and handling European events are a major issue for European countries. When citizens have misperceptions and nation-serving bias in memorizing and handling European events, citizens are likely to demand policies that they would not demand when being suitably informed. Platforms of established political parties converged in many European countries. Established political parties did not wish to offer polarized in dealing with the public debt crisis. Many citizens were disenchanted and, in turn, new populist political parties entered the political arena. Clearly, citizens' views about the public debt crisis and their perceptions about European policies gave rise to electoral success of new (populist) parties and, in turn, reinforced policies in Europe. 

Our study shows divergences in how European policies between European countries are perceived. Moreover, experts who advise politicians do not tend to have misperceptions about the European public debt crisis. Misperceptions among non-experts but no misperceptions among experts and policy-makers may well explain why European policies and the EU as an institution is considered as unpopular in many countries. 

We believe that (social) media influence citizens' views a great deal. An avenue for future research is examining the extent to which social media influence misperceptions and nation-serving bias. 
\clearpage
\section{Appendix}
\subsection{Survey Structure}
Our survey consists of two types of questions. We ask participants to either evaluate statements we make about the European debt crisis or ask to which party a certain statement applies. We further ask participants regarding their knowledge about the European debt crisis and socioeconomic characteristics. Our survey is structured in the following way. Our first question consists of asking people whether a country from the Eurozone received financial assistance. Participants can choose between three options: Yes, No and I don't know. Our second question asks about the beginning of the financial rescue program. 
Participants can indicate whether they strongly agree, slightly agree, slightly disagree or strongly disagree with or don't know about the following statements. 
The lender countries wanted to help the borrower countries. The lender countries wanted to help avoid a crisis at home. The lender countries wanted to impose institutional change upon the borrower countries. 
In the third question participants are asked "Who was the driving force behind the rescue program". In the fourth question they are asked "Which party benefited most from the rescue program" for both questions participants have the answer options: The lender countries, both parties equally, the borrower countries and I don' know. In the fifth question participants are asked about the feelings the rescue program evoked among the citizens from the borrower countries and the lender countries. As with question two participants have the option to answer with strongly agree, slightly agree, slightly disagree and strongly disagree or the option I don't know. The statements participants are asked to assess are the following: The rescue experience made citizens from the borrower countries feel guilty, the rescue experience made citizens from the borrower countries feel exploited, the rescue experience made citizens from borrower countries feel inferior, the rescue experience made citizens from lender countries feel exploited, rescue experience made citizens from the lender countries feel disappointed and the rescue experience strengthened friendships. In the last two questions of the survey participants are asked about the situation in Greece. The sixth question of our survey asks participants which party benefited most from loans to Greece. The answer options include Greece, the lender countries, both equally and I don't know. The last question of our survey asks participants whether Greece will repay it's debt. Participants can answer strongly agree, slightly agree, slightly disagree, strongly disagree and I don't know. Participants from both samples are asked the same questions. However, participants of the World Economic Survey are interviewed on a regular basis. Their survey further included questions on other macroeconomic variables. The survey was distributed to the participants on paper, while the participants recruited through the website prolific answered the survey online. In a comparison of the results from online and pen-and paper survey outcomes \cite{abel} finds no differences in the responses of participants. 


\subsection{Heterogeneity Analysis}
\textbf{Socioeconomic Characteristics}

\begin{figure} [h!]
    \begin{center}
     \caption{ Intentions of the borrower countries}
    \includegraphics[scale=1.2]{socio_question2.pdf}
    \label{fig:my_label}
    \end{center}
    \tiny 
    \tablenotes{Participants were asked to assess the following statements:  Question 2.1: The lender countries wanted to help the borrowing countries Question 2.2: The lender countries wanted to help themselves avoid a crisis at home Question 2.3: The lender countries wanted to impose institutional change upon the borrower countries }
\end{figure}
\begin{figure}[h!]
    \begin{center}
     \caption{ Emotions of program countries}
    \includegraphics[scale=1.2]{socio_question5_1.pdf}
    \label{fig:my_label}
    \end{center}
    \tiny 
     \tablenotes{Question 5.1: The rescue experience made many citizens in the borrower countries feel guilty; Question 5.2: The rescue experience made many citizens in the borrower countries feel exploited; Question 5.3: The rescue experience made many citizens in the borrower countries feel inferior} 
\end{figure}
\begin{figure}[h!]
    \begin{center}
     \caption{ Emotions of non-program countries and repayment of outstanding debt}
    \includegraphics[scale=1.2]{socio_question5_2.pdf}
    \end{center}
    \tiny
     \tablenotes{Question 5.4: The rescue experience made many citizens in the lender countries feel exploited; Question 5.5 The rescue experience made many citizens in the lender countries feel disappointed Question 5.6: The rescue experience strengthened friendships between citizens Question 7: Greece will fully pay back it's debt}
\end{figure}
\begin{figure}[h!]
    \begin{center}
     \caption{Who initiated and benefited from the rescue program}
    \includegraphics[scale=0.9]{socio_question3_4.pdf}
    \end{center}
    \tiny
    \tablenotes{Question 3: Who was the driving force behind signing the memorandum; Question 4: Who was the main beneficiary of the program; Question 7: Who primarily benefited from the loans to Greece}
\end{figure}
\begin{figure}[h!]
    \begin{center}
     \caption{Situation in Greece}
    \includegraphics[scale=0.9]{socio_question6_7.pdf}
    \end{center}
    \tiny
    \tablenotes{Question 6: Who primarily benefited from the loans to Greece? ;  Question 7: Greece will fully pay back it's debt}
\end{figure}
\clearpage


\textbf{Knowledge and Beliefs}
%articipants differ in their level of knowledge about the European debt crisis. Some people fail to correctly identify their country as a borrower or lender country. An overview of the fraction of participants who knew their country's status can be found in the appendix. However, knowledge about the status of one's country does not appear to influence the observed effects. The estimates of the baseline model on the subsample of non-experts who could correctly identify their country does not change in comparison to the estimates for the full sample. 

\\\\
%We redefine the program variable according to the beliefs of the survey participants. We now estimate the divergence in answers between participants which believed to be the national of a  lender country and participants which believed to be the national of a borrower country. Replacing the program variable by beliefs about belonging to a program country yields different results than the baseline model. In comparison to participants who believe to be lenders, participants who believe to be borrowers do not agree less that lender countries wanted to help borrower countries. They also do not agree more that the rescue program made citizens in the borrower countries feel guilty or inferior and or less likely to state that the lender countries were the driving force behind signing the referendum. Interestingly, differences emerge in the agreement about the feelings of citizens in the lender countries. Participants who believe they live in borrower countries are less likely to agree that citizens in the lender countries felt exploited or disappointed. They are further more likely to believe that the rescue program strengthened friendships between citizens. \\

\\
%Our heterogeneity analysis yields some interesting observations about potential drivers of the observed differences between expert and non-expert sample. The comparison of means suggests that the observed differences are not driven by the opinion of experts resembling the opinion of non-experts from either program or non-program countries. 
%Our analysis suggests that the observed difference between experts and non-experts cannot be explained by differential effects across age or education levels. However, it appears that non-experts which are more mobile do not show a strong nation-serving bias in their assessments of the European debt crisis. Interestingly, being able to correctly identify one's country as a program or a lender country does not change the observed magnitude of results. The magnitude of effects does change however, when redefining the program variable according to beliefs of people. This suggests that collective memory might work on a more subconscious level regardless of the level of information participants have. 
\\
\begin{figure} [h!]
    \begin{center}
     \caption{ Intentions of the borrower countries (Beliefs)}
    \includegraphics[scale=1.2]{belief_question2.pdf}
    \label{fig:my_label}
    \end{center}
    \tiny 
    \tablenotes{Participants were asked to assess the following statements:  Question 2.1: The lender countries wanted to help the borrowing countries Question 2.2: The lender countries wanted to help themselves avoid a crisis at home Question 2.3: The lender countries wanted to impose institutional change upon the borrower countries }
\end{figure}
\begin{figure}[h!]
    \begin{center}
     \caption{ Emotions of program countries (Beliefs) }
    \includegraphics[scale=1.2]{belief_question51.pdf}
    \label{fig:my_label}
    \end{center}
    \tiny 
     \tablenotes{Question 5.1: The rescue experience made many citizens in the borrower countries feel guilty; Question 5.2: The rescue experience made many citizens in the borrower countries feel exploited; Question 5.3: The rescue experience made many citizens in the borrower countries feel inferior} 
\end{figure}
\begin{figure}[h!]
    \begin{center}
     \caption{ Emotions of non-program countries and repayment of outstanding debt (Beliefs)}
    \includegraphics[scale=1.2]{belief_question52.pdf}
    \end{center}
    \tiny
     \tablenotes{Question 5.4: The rescue experience made many citizens in the lender countries feel exploited; Question 5.5 The rescue experience made many citizens in the lender countries feel disappointed Question 5.6: The rescue experience strengthened friendships between citizens Question 7: Greece will fully pay back it's debt}
\end{figure}
\begin{figure}[h!]
    \begin{center}
     \caption{Who initiated and benefited from the rescue program (Beliefs) }
    \includegraphics[scale=1.2]{belief_question34.pdf}
    \end{center}
    \tiny
    \tablenotes{Question 3: Who was the driving force behind signing the memorandum; Question 4: Who was the main beneficiary of the program; Question 7: Who primarily benefited from the loans to Greece}
\end{figure}
\begin{figure}[h!]
    \begin{center}
     \caption{Situation in Greece (Beliefs)}
    \includegraphics[scale=1.2]{belief_question67.pdf}
    \end{center}
    \tiny
    \tablenotes{Question 6: Who primarily benefited from the loans to Greece? ;  Question 7: Greece will fully pay back it's debt}
\end{figure}














\clearpage
\bibliographystyle{apalike}
\bibliography{bibliography.bib}
\end{document}
