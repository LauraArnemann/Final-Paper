\section{Descriptive Statistics}

\textbf{Comparison of Means}
We compare the mean answers of the experts with the mean answers of non-experts from program and non-program countries to determine whether experts side more with the opinion of borrower or lender countries.\\
When asked about the intentions of the lender countries experts do not show a tendency to side with neither non-experts from program countries or non-experts from non-program countries. They are slightly more likely to agree that lender countries wanted to help borrower countries than non-experts, more along the lines of non-experts from non-program countries. Further, they are more likely to agree that lender countries engaged in the rescue program out of self-interest siding with the borrower countries on this issue. The lack of agreeing with either side also emerges when expert are asked about the emotions the rescue program evoked among citizens from borrower and lender countries respectively. Overall, experts agree with the statements that citizens in borrower countries felt exploited and inferior more than non-experts. Hence, on this issue the opinion of experts resembles the opinion of non-experts from program countries. However, they also agree more to the statement that citizens in lender countries felt disappointed than non-experts. Experts from program and non-program countries only show statistically significant different level of agreements with the statement that Greece will fully pay back it's debt. For this question the average level of agreement in the expert and non-expert sample are similar for participants from program and non-program countries. \\
One interesting result emerges when experts are asked to assess to which party a certain statement applies. The answers from experts show some divergence on this matter, but in an opposite direction than non-experts. Experts from non-program countries are more likely to agree that their countries, the lender countries, were the main beneficiaries of the rescue program and loans to Greece. Further, experts from non-program countries are also more likely to agree than experts from program countries that lender countries were the driving force behind signing the referendum. However, it is important to note that these differences are not significantly different from zero.  \\

\textbf{Comparison of program countries} 
We compare the responses from participants of different program countries of the non-expert sample. We find that even within program countries survey participants in our non-expert sample diverge in their assessment of the European debt crisis. When asked about the intentions behind the rescue program it appears that Greek citizens show the strongest difference with respect to participants from non-program countries. Compared with the average assessment of program countries Greek participants are 20 percentage points less likely to agree that the lender countries wanted to help the borrowing countries, but strongly agree that the lender countries wanted to avoid a crisis at home and impose institutional change upon the borrower countries. In comparison to Greek participants other participants answer in a more moderate way. Notably, participants from Ireland are even 19 percentage points less likely to agree that lender countries wanted to impose institutional change than the average of program countries. The deviation of Greek participants is reinforced throughout the remaining survey questions. Participants from Greece agreed more strongly than participants from other program countries that the rescue program evoked feelings of guilt, exploitation and inferiority among citizens from the borrower countries (Question 5.1, 5.2, 5.3). Further, they are 14 and 16 percentage points less likely than citizens from program and non-program countries to agree that the rescue program strengthened friendships (Question(5.6). Contrary to our hypotheses Greek participants are also more likely to agree that the rescue program evoked negative feelings among the lender countries as well compared to the average of program countries (Question 5.4, 5.5).  They are more likely to agree that Greece will be able to pay it's debt, however the effect is not very large (Question 6). Greek participants strongly agree that the lender countries were the main beneficiaries of the loans to program countries and Greece and were also the main driving force behind initiating the rescue program (Questions 3, 4, 7).  All other program countries, which have already succeeded in repaying their debt show a weaker self-serving bias than Greek citizens. 
\\
Due to the small size of our expert sample we refrain from running inter-country comparisons. 


 \begin{table}[h!]
\caption{ Comparison of Means between individual program country and sample means} 
\resizebox{\textwidth}{!}{%
\begin{tabular}{*{7}{>{\centering\arraybackslash}p{.13\linewidth}}}
\hline\hline
&\multicolumn{4}{c}{\textbf{Deviation from the Program Country Mean}} &\multicolumn{2}{c}{\textbf{Means}} \\
           Question &\multicolumn{1}{c}{Greece}&\multicolumn{1}{c}{Ireland}&\multicolumn{1}{c}{Portugal}&\multicolumn{1}{c}{Spain}&\multicolumn{1}{c}{Program (mean)}&\multicolumn{1}{c}{Non-Program (mean) }\\
  
\hline
2.1 (-)       &       -0.20&        0.11&       -0.04&        0.03&        0.51&        0.62\\
2.2 (+)        &        0.05&        0.01&       -0.03&       -0.06&        0.91&        0.90\\
2.3 (+)       &        0.13&       -0.19&        0.03&       -0.08&        0.74&        0.62\\
\hline
&&&&&& \\
5.1 (+)        &        0.11&        0.04&       -0.05&       -0.02&        0.47&        0.39\\
5.2 (+)        &        0.14&       -0.05&       -0.01&       -0.16&        0.79&        0.61\\
5.3 (+)        &        0.10&        0.03&       -0.10&       -0.04&        0.72&        0.62\\
5.4 (-)        &        0.13&       -0.10&       -0.03&       -0.04&        0.60&        0.60\\
5.5 (-)        &        0.08&       -0.18&       -0.03&       -0.04&        0.61&        0.61\\
5.6  ()       &       -0.14&        0.10&        0.05&        0.02&        0.25&        0.27\\
\hline
&&&&&& \\
6  (+)         &        0.04&       -0.02&       -0.02&        0.03&        0.32&        0.18\\
3  (+)          &        0.14&        0.03&       -0.12&       -0.03&        0.54&        0.43\\
4  (+)          &        0.17&        0.03&        0.05&       -0.17&        0.67&        0.44\\
7  (+)          &        0.31&       -0.11&       -0.06&       -0.12&        0.55&        0.36\\
\hline\hline

\end{tabular} }
\begin{tablenotes}
\footnotesize
\item The sign in parantheses denotes the expectation about the difference in assessments between program and non-program countries. The left-hand side illustrates the difference between single program countries and the overall program country mean. The right hand side shows the program country mean and the non-program country mean. 
\end{tablenotes}
\end{table}

