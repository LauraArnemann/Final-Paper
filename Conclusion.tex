\section{Conclusion} 
The 2010 European public debt crisis influenced policies, politics and voters' perceptions. For example, domestic policy-makers introduced measures such as fiscal rules to handle increasing public debt and budget deficits. The European Central Bank pursued expansionary monetary policies: it introduced the Outright Monetary Transactions (OMT) program and decreased interest rates to zero. Rescue programs for Cyprus, Greece, Ireland, Portugal and Spain were designed. Platforms of established political parties converged in many European countries. Established political parties did not wish to offer polarized in dealing with the public debt crisis. Many citizens were disenchanted and, in turn, new populist political parties entered the political arena. Clearly, citizens' views about the public debt crisis and their perceptions about European policies gave rise to electoral success of new (populist) parties and, in turn, reinforced policies in Europe. 

We examined citizens' views about the European public debt crisis. In particular, we investigated collective memory of the European public debt crisis and disentangled views from citizens from rescue program countries and citizens from non-program countries. During the public debt crisis, media reports have suggested that citizens from program and non-program countries have different views on the crisis and how to handle it. An important question is whether collective memories between citizens from program and non-program also differ. Studies in psychology suggest that individual borrowers and lenders remember credit relationships in different manners (\cite{dezso}): memories are influenced by a self-serving bias. It is conceivable that collective memory about credit-relationships also give rise to a nation-serving bias. Examining a nations-serving bias in collective memory is new.

We compiled new data measuring collective memories on the European debt crisis. We asked economic experts by using CESifo's World Economic Survey and non-experts by using the provider prolific. The results suggest that experts from lender and borrower countries have quite similar views about the European public debt crisis. The views of non-experts are, by contrast, influenced by a nation-serving bias. The bias relates to memories on which countries signed a memorandum, why they signed it, and how they assess the measures taken to address the crisis. These results corroborate empirical studies about citizens' misperceptions about macroeconomic policies and outcomes. Citizens evaluate macroeconomic policies and outcomes much better when their preferred political party is in office than when parties govern that they do not support (on partisan bias see, for example, \cite{evans}, \cite{gerber}, \cite{gillitzer}, \cite{bachmann}). 

We do not observe any nations-serving bias among non-experts when we examine non-experts that left their country of birth. Cosmopolitans are less likely to evaluate European policies with a nation-serving bias than non-Cosmopolitans (see also \cite{bechtel}).

We believe that citizens' misperceptions and nation-serving bias in memorizing and handling European events are a major issue for industrialized countries. When citizens have misperceptions and nation-serving bias in memorizing and handling European events, citizens are likely to demand policies that they would not demand when being suitably informed. Election-motivated politicians are, in turn, inclined to offer to policies citizens demand - the desired policies being influenced by miscperceptions and nation-serving bias notwithstanding. 

We believe that (social) media influence citizens' views a great deal. An avenue for future research is examining the extent to which social media influence misperceptions and nation-serving bias. 