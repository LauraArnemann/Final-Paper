\documentclass[12pt]{article}
\usepackage{array}
\usepackage{graphicx} % Allows including images
\usepackage{tabularx}

\usepackage{float}

\newtheorem{theorem}{Theorem}
\newtheorem{corollary}[theorem]{Corollary}
\newtheorem{proposition}{Proposition}
\newenvironment{proof}[1][Proof]{\noindent\textbf{#1.} }{\ \rule{0.5em}{0.5em}}

\newtheorem{hyp}{Hypothesis}
\newtheorem{subhyp}{Hypothesis}[hyp]
\renewcommand{\thesubhyp}{\thehyp\alph{subhyp}}

\geometry{left=1.0in,right=1.0in,top=1.0in,bottom=1.0in}

\begin{document}
\title{Divergence of Collective Memory on European Financial Entrenchment -
Implications for the European Union (working title)}
\author{Kai and Niklas}
\maketitle

\begin{abstract}
We study...
\end{abstract}
\clearpage
\section{Introduction}
This paper studies systematic differences in how citizens and experts from
different countries remember and examine the European sovereign debt crisis
that became salient in 2010. The crisis partitioned the countries inside the
Eurozone. Some Eurozone member states experienced fiscal and financial
conditions that dramatically worsened and were in danger of losing access to
the capital market for refinancing their public loans. The remaining
countries had to fear contagion effects and other negative spillovers of
possible national bankruptcies in the more severely affected countries.\
Five countries (Greece, Ireland, Portugal, Spain and Cyprus) applied for,
and signed a memorandum of understanding with the member counties of the
European Monetary Union. As part of these agreements, they received
financial aid or guarantees, but also accepted supervised mandatory
structural reforms. Simplifying and abbreviating these measures taken, we
refer to the program countries as the 'borrower countries'. On the other
side were the countries that used their fiscal credibility to provide these
guaranties and requested and participated in a monitoring of the process of
structural reforms. We refer to them as 'lender countries'.

The key question we ask is whether there are systematic differences in how
citizens and experts from borrower and lender countries remember the crisis
events. Would citizens from the two groups have similar memories about who
actually applied for a memorandum of understanding? How do they remember
whether the lender or the borrower countries pushed more for it and whether
the memoranda of understanding mainly benefited the lender countries or the
borrower countries? We also investigate how these programs were perceived in the
respective countries and how these crisis intervention affected the
relationship between European countries more generally.\ 

To study these questions we first considered possible divergence in the
views of experts. For this purpose we included specific questions
to the World Economic Survey (WES) panel of experts. Then we paid a larger
number of members of the general population in countries in Europe in an
internet questionnaire (Prolific) to answer the same set of questions. We
find no systemematic divergence in the answers by the WES experts from borrower and lender countries. However, significant differences emerge in how
the citizens from the two groups of countries remember what countries signed
a memorandum, why they signed it, and how they assess the measures taken to
address the crisis.  

The empirical analysis of divergence of memories along the lines of borrower
countries and lender countries is motivated by several blocks of
theory. An important theory about memory formation has been developed in
psychology, neuroscience and sociology. It combines insights about the
plasticity of memory\footnote{%
The physiological basics for how memory is formed, kept and reactivated have
been explored. As Dudai and Edelson (2016;\ 276)\ describe, "When the memory
is retrieved, it seems to re-enter a transient phase in which it again
becomes susceptible to the same amnesic agents that were effective in the
original consolidation window (Dudai, 2012; Nader et al., 2000; Sara,
2000)." Brain sciences hence suggest that memory enters a state of
plasticity when it is reactivated and the copy that is then stored might
differ from the one that has been activated (see Agren 2014, and Lee, Nader
and Schiller 2017).}, and a tendency to memorize in a 'self-serving' way
(see Bell and Echterhoff 2014). If this self-serving bias affects the
process of reactivating, transforming and storing memory, it can cause
memory with a self-serving drift.

The combination of plasticity of memory and the self-serving drift has been
applied by Desz\"{o} and Loewenstein (2012) to hypothesize self-serving
memories on informal credit relationships. Using data about informal credit
relationships between relatives and friends, they found that borrowers and
lenders diverge in their memories about these credit relationships in
divergent ways that appear to be self-serving.\footnote{%
Borrowers and lenders partially diverge in how they recall the credit event
as such, the conditions of the loan they agreed upon, and the mutual
interactions that made the contract come about. This divergence typically
has a self-serving bias. For instance, compared to the lenders, borrowers
tend to recall less if they exerted moral pressure to receive the loan and
are more likely to recall that the credit was generously offered to them.}
The lenders and borrowers in the European souvereign debt crisis are
nations, rather than individuals, but the line of reasoning could be
similar: when analysing the recollections on fiscal credit relationships or
loan guarantees between nations the mental processes of the formation,
storage and reactivation of memory remain individuals' activities. They
might follow the same logic and pyhsiological laws as in the context of
private borrowing-lending relationships. In the aggregate, individuals in
borrower-countries and individuals in lender-countries might recall
differently which side was the driving force for international credit
relationships: the lender-side interpretation might more often be that the
borrower country was desperately seeking help, whereas the borrower side
might tend to identify reasons why the lender-side forced the program upon
the borrower country. The two groups might also differ in their assessments
about who eventually benefited more from the conditional financial help
program.\footnote{%
A large literature discusses collective memory, in difference to the memory
of individuals. Olick (1999) provides a lucid survey about the origins and
meanings of the concept of collective memory in the early work by Halbwachs
(1925;\ 19xx) and highlights the double meaning of the term: `collected
memory' as some kind of aggregate of individual memories, and a
`collectivist' as compared to individualistic concept of something that the
members of a group such as a nation have in common and that is an important
tool to generate a common identity. If collective memory is just the
aggregated sum of memories of a given group, the mechanisms of self-serving
bias and transformation of memory through activation might be at work much
like for individual memory.}

Diverging views and memories, if they exist, might not only be an outcome of
such individual processes of self-serving memory biases. Biased public news
and newspaper reports that differ between countries could play an important
role. If the news reporting during the crisis and in the years that followed
had national biases that also went along the dividing line between borrower
countries an lender countries, this might contribute to a divergence in
memories. In combination with the discussed plasticity of memory in the
phase of reactivation, divergent media news might establish a channel that
strenghthens the self-serving bias:\ such media news activate memories, and
might transform the memory before it is stored again. 

The causal relationship between effects on the individual citizen
level and the media need not be unidirectional. It might  interact and
compound: common institutions inside a nation, such as common exposure to
the same public media and other public institutions might intensify
information exchange inside the group, might cause a continuous
transformation of this aggregate and might even strengthen and homogenize
the national collections of memories. For discussions see Rigney (2018) and
Roediger and Abel (2015;\ 361) who conclude: "Such collective memories
probably boost group identity and shape social and political discourse. In
particular, studies of how various groups remember `the same' events so
differently may help to uncover important psychological factors at work in
group dynamics and conflict." 

As we have data from 2009/2010 (experts/general population), we can address
the question whether the borrower or lender position are relevant for
differences in how the crisis is remembered and assessed. As we do not have
panel data or time variation, we will not be able to assess the dynamics of
differences in collective memories. However, the distinction between experts
and regular citizens can be helpful. 
\section{The Design}
\subsection{The Sample}
Our data was obtained through a survey we conducted among two pools of participants. In August 2018, we asked economic experts from the World Economic Survey (WES) for their opinion on perceptions about the financial entrenchment following the European debt crisis. The WES is a quarterly survey conducted by the ifo Institute. The survey includes many  questions, indicating the opinion towards overall economic development from European and non-European  experts such as economic growth and inflation. The WES also includes special questions on current economic issues. Our WES sample includes 517 participants from EU member countries, among them 90 from program countries.
\\
Our second pool of participants was recruited through the website prolific.co.In contrast to other crowdsourcing platforms such as Mturk Prolific is a platform specifically designed to recruit participants for academic research \footnote{\cite{Peer} demonstrate that participants from the platform prolific perform better than participants from other crowdsourcing platforms}. In exchange for their participation in surveys or experiments, participants receive a financial reward. The survey was distributed to 1702 participants in August 2019, 498 of these participants came from program countries. To ensure that our participants actively remember the events during the European debt crisis we restrict our sample to include only participants older than 25. \\
Participants from both samples received the survey questions in the same ordering. The appendix shows the origin of survey participants from both samples.  
\subsection{Sample Characteristics} 
The WES sample and the prolific sample differ along various dimensions. More than 80 percent of WES participants are male, whereas in the prolific sample there is an equal share of men and women. The majority of participants from the prolific sample, around 65 percent are younger than 35, whereas the majority of participants from the WES sample are between 35 and 55. Participants from the WES sample are also have a higher level of education than participants from the prolific sample, 60 percent of participants hold a PhD. Nonetheless the majority of participants from the prolific sample have completed tertiary education. We henceforth refer to the WES sample as the expert sample and to the prolific sample as the non-expert sample. 
\subsection{Survey Structure}
Our survey contains seven main questions about participants recollection of various aspects of the European debt crisis consisting of several  We base our questions on the survey conducted by \cite{dezso}. We use two types of questions to examine whether citizens from borrower countries perceive the European debt crisis in a different manner than citizens from from lender countries. First, we ask experts to assess their level of agreement with a certain statement on a scale of 1 to 4. Second we ask participants to name a party to which a certain statement applies. Participants can name the borrower countries, the lender countries our both equally. We further ask participants which countries participated in the European debt crisis and collect information on socioeconomic characteristics. We ask participants from the prolific sample and the WES sample to indicate their level of education, age, gender. Participants from the prolific sample are also asked to name their employment status, WES participants to name their affiliation. 

\section{The  Hypotheses}
We expect our participants to show a nation-serving bias in their answers to our survey questions. In the following we predict how the answers of participants from program countries will differ with regard to the answers of participants from non-program countries. The full set of survey questions is shown in the appendix. 


\begin{enumerate} 
\item\textbf{Why the lender countries engaged in the credit relationship} \\
We ask participants whether lender countries acted out of benevolence (to help the borrower countries) or out of self-interest (to avoid a crisis in their own countries/to force institutional change upon the borrower countries). We predict that participants from program countries are more likely to state that lender countries acted out of self-interest and less likely that they acted out of benevolence. 

\item \textbf{Who initiated the credit relationship} \\
We ask our survey participants to assess which party was the driving force behind the credit relationship. We predict that participants from program countries will be more likely to state that citizens from lender countries initiated the credit relationship. 

\item \textbf{Which party benefited from the rescue program}\\ 
We first ask participants whether borrower or lender countries were the main beneficiaries of the rescue program. In a subsequent question we also ask whether Greece was the main beneficiary from the rescue program. For both answers we expect that participants from program countries will be more likely to state that lender countries were the main beneficiaries from the rescue program. 

\item \textbf{ What feelings the rescue program evoked among citizens from program and non-program countries}\\
Our participants are asked whether the rescue program evoked negative feelings among citizens from borrower countries (feeling guilty, exploited or inferior). Further, they are asked whether the rescue program evoked negative feelings among the lender countries (feelings of exploitation and disappointment).  
We expect participants from program countries to be more likely to agree that the rescue program evoked negative feelings among citizens from borrower countries and less likely to state that the rescue program evoked negative feelings among citizens from non-program countries. 
\item \textbf{Whether outstanding debt will be repaid} \\
We ask participants if Greece will be able to repay it's outstanding debt. \footnote{Greece is the only country which has not repaid the loans it received in the course of the rescue program} We expect citizens from program countries to demonstrate more confidence in Greece's ability to repay outstanding debt. 
\end{enumerate}

\section{Descriptive Statistics}

\textbf{Comparison of Means}
We compare the mean answers of the experts with the mean answers of non-experts from program and non-program countries to determine whether experts side more with the opinion of borrower or lender countries.\\
When asked about the intentions of the lender countries experts do not show a tendency to side with neither non-experts from program countries or non-experts from non-program countries. They are slightly more likely to agree that lender countries wanted to help borrower countries than non-experts, more along the lines of non-experts from non-program countries. Further, they are more likely to agree that lender countries engaged in the rescue program out of self-interest siding with the borrower countries on this issue. The lack of agreeing with either side also emerges when expert are asked about the emotions the rescue program evoked among citizens from borrower and lender countries respectively. Overall, experts agree with the statements that citizens in borrower countries felt exploited and inferior more than non-experts. Hence, on this issue the opinion of experts resembles the opinion of non-experts from program countries. However, they also agree more to the statement that citizens in lender countries felt disappointed than non-experts. Experts from program and non-program countries only show statistically significant different level of agreements with the statement that Greece will fully pay back it's debt. For this question the average level of agreement in the expert and non-expert sample are similar for participants from program and non-program countries. \\
One interesting result emerges when experts are asked to assess to which party a certain statement applies. The answers from experts show some divergence on this matter, but in an opposite direction than non-experts. Experts from non-program countries are more likely to agree that their countries, the lender countries, were the main beneficiaries of the rescue program and loans to Greece. Further, experts from non-program countries are also more likely to agree than experts from program countries that lender countries were the driving force behind signing the referendum. However, it is important to note that these differences are not significantly different from zero.  \\

\textbf{Comparison of program countries} 
We compare the responses from participants of different program countries of the non-expert sample. We find that even within program countries survey participants in our non-expert sample diverge in their assessment of the European debt crisis. When asked about the intentions behind the rescue program it appears that Greek citizens show the strongest difference with respect to participants from non-program countries. Compared with the average assessment of program countries Greek participants are 20 percentage points less likely to agree that the lender countries wanted to help the borrowing countries, but strongly agree that the lender countries wanted to avoid a crisis at home and impose institutional change upon the borrower countries. In comparison to Greek participants other participants answer in a more moderate way. Notably, participants from Ireland are even 19 percentage points less likely to agree that lender countries wanted to impose institutional change than the average of program countries. The deviation of Greek participants is reinforced throughout the remaining survey questions. Participants from Greece agreed more strongly than participants from other program countries that the rescue program evoked feelings of guilt, exploitation and inferiority among citizens from the borrower countries (Question 5.1, 5.2, 5.3). Further, they are 14 and 16 percentage points less likely than citizens from program and non-program countries to agree that the rescue program strengthened friendships (Question(5.6). Contrary to our hypotheses Greek participants are also more likely to agree that the rescue program evoked negative feelings among the lender countries as well compared to the average of program countries (Question 5.4, 5.5).  They are more likely to agree that Greece will be able to pay it's debt, however the effect is not very large (Question 6). Greek participants strongly agree that the lender countries were the main beneficiaries of the loans to program countries and Greece and were also the main driving force behind initiating the rescue program (Questions 3, 4, 7).  All other program countries, which have already succeeded in repaying their debt show a weaker self-serving bias than Greek citizens. 
\\
Due to the small size of our expert sample we refrain from running inter-country comparisons. 


 \begin{table}[h!]
\caption{ Comparison of Means between individual program country and sample means} 
\resizebox{\textwidth}{!}{%
\begin{tabular}{*{7}{>{\centering\arraybackslash}p{.13\linewidth}}}
\hline\hline
&\multicolumn{4}{c}{\textbf{Deviation from the Program Country Mean}} &\multicolumn{2}{c}{\textbf{Means}} \\
           Question &\multicolumn{1}{c}{Greece}&\multicolumn{1}{c}{Ireland}&\multicolumn{1}{c}{Portugal}&\multicolumn{1}{c}{Spain}&\multicolumn{1}{c}{Program (mean)}&\multicolumn{1}{c}{Non-Program (mean) }\\
  
\hline
2.1 (-)       &       -0.20&        0.11&       -0.04&        0.03&        0.51&        0.62\\
2.2 (+)        &        0.05&        0.01&       -0.03&       -0.06&        0.91&        0.90\\
2.3 (+)       &        0.13&       -0.19&        0.03&       -0.08&        0.74&        0.62\\
\hline
&&&&&& \\
5.1 (+)        &        0.11&        0.04&       -0.05&       -0.02&        0.47&        0.39\\
5.2 (+)        &        0.14&       -0.05&       -0.01&       -0.16&        0.79&        0.61\\
5.3 (+)        &        0.10&        0.03&       -0.10&       -0.04&        0.72&        0.62\\
5.4 (-)        &        0.13&       -0.10&       -0.03&       -0.04&        0.60&        0.60\\
5.5 (-)        &        0.08&       -0.18&       -0.03&       -0.04&        0.61&        0.61\\
5.6  ()       &       -0.14&        0.10&        0.05&        0.02&        0.25&        0.27\\
\hline
&&&&&& \\
6  (+)         &        0.04&       -0.02&       -0.02&        0.03&        0.32&        0.18\\
3  (+)          &        0.14&        0.03&       -0.12&       -0.03&        0.54&        0.43\\
4  (+)          &        0.17&        0.03&        0.05&       -0.17&        0.67&        0.44\\
7  (+)          &        0.31&       -0.11&       -0.06&       -0.12&        0.55&        0.36\\
\hline\hline

\end{tabular} }
\begin{tablenotes}
\footnotesize
\item The sign in parantheses denotes the expectation about the difference in assessments between program and non-program countries. The left-hand side illustrates the difference between single program countries and the overall program country mean. The right hand side shows the program country mean and the non-program country mean. 
\end{tablenotes}
\end{table}
\clearpage


\section{Heterogeneity Analysis and Robustness Checks}
Our results show that experts and non-experts show differences in their assessment of the European debt crisis. This finding is in line with work by \cite{roth} who using the WES and a representative sample show that experts differ from non-experts in their beliefs about the impacts of macroeconomic shocks. In our heterogeneity analysis we investigate which difference drives the observed results. We also examine whether individual factors influence the magnitude of the observed divergence. 
\\



\textbf{Socioeconomic Characteristics}
 Experts and non-experts differ along various dimensions such as education, age and gender.
 According to \cite{baumeister} events experienced at a younger age might have a more defining impact. Since the European debt crisis was accompanied for example by a high level of youth unemployment it might seem plausible that the divergences in memory might be more pronounced among the younger generation. Thus, we estimate the effect of belonging to a program country among participants older than 35 among the sample of non-experts.  The effect of the program variable on the likelihood to agree that the lender countries wanted to impose institutional change or that the rescue experience made citizens in the borrower countries feel guilty is smaller and the significance level decreases to 5 percent. For the remaining questions the overall magnitude and significance level of the program effect stays the same. 
\\
The level of education of participants may well influence the estimates. Participants with a higher level of education might have different political attitudes or consume and access different types of media. Hence, we split the non-expert sample and estimate our model only for participants reporting to have completed tertiary education. We do not find any change in the magnitude and significance level of effects for this subsample. 
\\
One dimension along which experts and non-experts differ is their degree of mobility. Experts working in think tanks or research institutes might live or have studied abroad for some time. Due to this circumstance experts might identify less with their nation than non-experts and consequently will not have a strong nation-serving bias. Unfortunately we don't have information about whether participants in the non-expert have lived or studied abroad. However, we can identify if people reported to be living in a different country than their country of birth. This applies to 25 percent of the non-expert subsample. 20.72 percent of participants from program countries and 25.45 percent from non-program countries report to be living in a different country than their country of birth. Estimating our model for this subsample changes the results quite a bit. Divergence between citizens from program- and non-program countries remains in the assessment if lender countries wanted to help borrowing countries and if the rescue experience made the citizens in the borrower countries feel exploited. For the other questions the difference in answers between program- and non-program countries becomes smaller and even vanishes completely for some questions. Hence, it appears that a higher level of mobility might be causal for the differential effects between expert and non-expert sample.  \\

\\
\textbf{Knowledge and beliefs} 
Participants differ in their level of knowledge about the European debt crisis. Some people fail to correctly identify their country as a borrower or lender country. An overview of the fraction of participants who knew their country's status can be found in the appendix. However, knowledge about the status of one's country does not appear to influence the observed effects. The estimates of the baseline model on the subsample of non-experts who could correctly identify their country does not change in comparison to the estimates for the full sample. 

\\\\
We redefine the program variable according to the beliefs of the survey participants. We now estimate the divergence in answers between participants which believed to be the national of a  lender country and participants which believed to be the national of a borrower country. Replacing the program variable by beliefs about belonging to a program country yields different results than the baseline model. In comparison to participants who believe to be lenders, participants who believe to be borrowers do not agree less that lender countries wanted to help borrower countries. They also do not agree more that the rescue program made citizens in the borrower countries feel guilty or inferior and or less likely to state that the lender countries were the driving force behind signing the referendum. Interestingly, differences emerge in the agreement about the feelings of citizens in the lender countries. Participants who believe they live in borrower countries are less likely to agree that citizens in the lender countries felt exploited or disappointed. They are further more likely to believe that the rescue program strengthened friendships between citizens. \\

\\
Our heterogeneity analysis yields some interesting observations about potential drivers of the observed differences between expert and non-expert sample. The comparison of means suggests that the observed differences are not driven by the opinion of experts resembling the opinion of non-experts from either program or non-program countries. 
Our analysis suggests that the observed difference between experts and non-experts cannot be explained by differential effects across age or education levels. However, it appears that non-experts which are more mobile do not show a strong nation-serving bias in their assessments of the European debt crisis. Interestingly, being able to correctly identify one's country as a program or a lender country does not change the observed magnitude of results. The magnitude of effects does change however, when redefining the program variable according to beliefs of people. This suggests that collective memory might work on a more subconscious level regardless of the level of information participants have. 
\\
\\
We also conduct various robustness checks. \\

\textbf{Ordered and Multinomial Estimation}
Participants have multiple answer possibilities. Depending on the question participants are able to rank their level of agreement on a scale of 1 to 4 or choose between the options borrower countries, lender countries or both equally. To control if differences between participants from non-program and program countries also emerge when including all answer possibilities as regression outcomes we estimate multinomial and ordered logit models. For all questions in which participants were asked to rank their level of agreement we estimate an ordered logit model, for questions in which participants were asked to name the responsible party we estimate an multinomial logit model. The significance level in the non-expert sample remains at the same level for all but one question. For the question which party was the driving force behind the referendum the significance level changes from 0.01 percent to 0.05 percent. In the expert sample all results remain insignificant. For the question if Greece will repay it's debt the divergence in effects is only significant at the 5 percent level and no longer the 1 percent level. In the majority of cases the the magnitude of effects is the same for different answer possibilities. This suggests that differences between program and non-program countries exist across all answer possibilities. \\

\\
\textbf{Inattentive Respondents}
Since we distributed our survey for the non-expert sample online we will also control for the influence of inattentive respondents in the non-expert sample. When distributing the survey online we already included an attention check when asking participants about socioeconomic characteristics. All participants failing this attention check were excluded from the survey. Further, we exclude all participants at the top 10 $\%$ and bottom 10 $\%$ of the survey time distribution. Excluding these participants does not change the inferences of our baseline estimation in the non-expert sample.\\


\\
\textbf{Clustered Robust Standard Errors} 
We cluster standard errors on the country level. Since we only collected data from 24 countries we adjust for the small number of clusters using the wild bootstrap method for logit regressions as suggested by \cite{cameron}.\footnote{We use the Stata command developed by \cite{roodman}} Inferences change for some questions when applying this method. Question 2.c becomes insignificant, questions 5a and 5c loose some significance and become significant at the 5 respectively 10 percent level.  \\

\textbf{Multiple Hypothesis Testing}
We also control for multiple hypothesis testing by adjusting our p-values using the Bonferroni Method. We adjust p-values by the number of questions we ask our participants. The Bonferroni correction does not change the significance level of our results. 

\section{Conclusion} 
\clearpage

\end{document}
