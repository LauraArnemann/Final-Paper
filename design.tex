\section{The Design}
\subsection{The Sample}
Our data was obtained through a survey we conducted among two pools of participants. In August 2018, we asked economic experts from the World Economic Survey (WES) for their opinion on perceptions about the financial entrenchment following the European debt crisis. The WES is a quarterly survey conducted by the ifo Institute. The survey includes many  questions, indicating the opinion towards overall economic development from European and non-European  experts such as economic growth and inflation. The WES also includes special questions on current economic issues. Our WES sample includes 517 participants from EU member countries, among them 90 from program countries.
\\
Our second pool of participants was recruited through the website prolific.co.In contrast to other crowdsourcing platforms such as Mturk Prolific is a platform specifically designed to recruit participants for academic research \footnote{\cite{Peer} demonstrate that participants from the platform prolific perform better than participants from other crowdsourcing platforms}. In exchange for their participation in surveys or experiments, participants receive a financial reward. The survey was distributed to 1702 participants in August 2019, 498 of these participants came from program countries. To ensure that our participants actively remember the events during the European debt crisis we restrict our sample to include only participants older than 25. \\
Participants from both samples received the survey questions in the same ordering. The appendix shows the origin of survey participants from both samples.  
\subsection{Sample Characteristics} 
The WES sample and the prolific sample differ along various dimensions. More than 80 percent of WES participants are male, whereas in the prolific sample there is an equal share of men and women. The majority of participants from the prolific sample, around 65 percent are younger than 35, whereas the majority of participants from the WES sample are between 35 and 55. Participants from the WES sample are also have a higher level of education than participants from the prolific sample, 60 percent of participants hold a PhD. Nonetheless the majority of participants from the prolific sample have completed tertiary education. We henceforth refer to the WES sample as the expert sample and to the prolific sample as the non-expert sample. 
\subsection{Survey Structure}
Our survey contains seven main questions about participants recollection of various aspects of the European debt crisis consisting of several  We base our questions on the survey conducted by \cite{dezso}. We use two types of questions to examine whether citizens from borrower countries perceive the European debt crisis in a different manner than citizens from from lender countries. First, we ask experts to assess their level of agreement with a certain statement on a scale of 1 to 4. Second we ask participants to name a party to which a certain statement applies. Participants can name the borrower countries, the lender countries our both equally. We further ask participants which countries participated in the European debt crisis and collect information on socioeconomic characteristics. We ask participants from the prolific sample and the WES sample to indicate their level of education, age, gender. Participants from the prolific sample are also asked to name their employment status, WES participants to name their affiliation. 

\section{The  Hypotheses}
We expect our participants to show a nation-serving bias in their answers to our survey questions. In the following we predict how the answers of participants from program countries will differ with regard to the answers of participants from non-program countries. The full set of survey questions is shown in the appendix. 


\begin{enumerate} 
\item\textbf{Why the lender countries engaged in the credit relationship} \\
We ask participants whether lender countries acted out of benevolence (to help the borrower countries) or out of self-interest (to avoid a crisis in their own countries/to force institutional change upon the borrower countries). We predict that participants from program countries are more likely to state that lender countries acted out of self-interest and less likely that they acted out of benevolence. 

\item \textbf{Who initiated the credit relationship} \\
We ask our survey participants to assess which party was the driving force behind the credit relationship. We predict that participants from program countries will be more likely to state that citizens from lender countries initiated the credit relationship. 

\item \textbf{Which party benefited from the rescue program}\\ 
We first ask participants whether borrower or lender countries were the main beneficiaries of the rescue program. In a subsequent question we also ask whether Greece was the main beneficiary from the rescue program. For both answers we expect that participants from program countries will be more likely to state that lender countries were the main beneficiaries from the rescue program. 

\item \textbf{ What feelings the rescue program evoked among citizens from program and non-program countries}\\
Our participants are asked whether the rescue program evoked negative feelings among citizens from borrower countries (feeling guilty, exploited or inferior). Further, they are asked whether the rescue program evoked negative feelings among the lender countries (feelings of exploitation and disappointment).  
We expect participants from program countries to be more likely to agree that the rescue program evoked negative feelings among citizens from borrower countries and less likely to state that the rescue program evoked negative feelings among citizens from non-program countries. 
\item \textbf{Whether outstanding debt will be repaid} \\
We ask participants if Greece will be able to repay it's outstanding debt. \footnote{Greece is the only country which has not repaid the loans it received in the course of the rescue program} We expect citizens from program countries to demonstrate more confidence in Greece's ability to repay outstanding debt. 
\end{enumerate}
