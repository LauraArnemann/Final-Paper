\section{Robustness Checks}

Borrower and lender countries vary along other dimensions than the borrower/ lender distinction. Program countries share several characteristics which are likely to influence the estimates. Countries which were affected by the European debt crisis are predominantly Southern European and are located at the periphery of the European Union (measured by distance to Brussels). Further, all these countries experienced a substantial increase in debt levels, high levels of unemployment, low GDP growth and belonged to the Eurozone. Thus we conduct sample splits along these margins to test whether these variables drive our results. \footnote{ We again estimate our model on the subsample of countries which are Southern European, belong to the Eurozone,are located at the periphery of Europe,defined as countries in which the distance of the capital to Brussels is above the median, experienced above median debt and unemployment growth and below median GDP growth during the years 2007 and 2012.} The overall findings of our analysis remain unchanged among the expert and non-expert sample.
\\
In addition to controlling for the influence of macroeconomic variables we also conduct several other robustness checks. We estimate the model by ordered and multinomial estimation, we cluster standard errors at the country level, control for multiple hypothesis testing and drop inattentive respondents from our non-expert sample. All these checks do not yield substantially different results than our baseline estimation. A detailed overview of the results of our heterogeneity analysis and robustness checks can be found in the appendix. 

